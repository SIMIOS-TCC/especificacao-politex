\chapter{Especificação de Requisitos de Sistema}
O funcionamento essencial do sistema, o que define seus requisitos funcionais, requeriu que a posição dos macacos seja possível de ser medida, armazenada e mostrada para o usuário.

Além desses, foram levantados os requisitos não funcionais, que trabalham aspectos necessários e complementares para o bom funcionamento do sistema, muitas vezes previstos pelo público solicitante do mesmo.

No SIMIOS, os principais requisitos não funcionais foram apontados por pesquisadores biólogos e veterinários com experiência em monitoramento de macacos. Dentre eles, está que o peso da mochila que será anexada ao animal não deveria ultrapassar 10g para não influenciar em seu comportamento nem sobrecarregá-lo, visto que o principal grupo de foco (saguis) tem peso médio de 400g. Para isso, era interessante que todos os componentes da mochila fossem o mais leves possível.

Outra situação apontada é o fato de que toda vez que a bateria do aparelho tiver de ser trocada, o veterinário deverá capturar o macaco e sedá-lo, o que é bastante prejudicial para a confiança que o animal constrói pelo ser humano. Dessa forma, era desejável que a eficiência energética do dispositivo embarcado seja alta para que a bateria tenha de ser trocada com a menor frequência possível.

Além da mochila do animal, os dados coletados deveriam ser confiáveis. Isso envolve garantir a validade das medidas enviadas e a minimização de seus erros. Assim, remedia-se casos nos quais há medidas falsas, por exemplo, quando o dispositivo é removido acidentalmente do animal, bem como nos quais haja interferências ruidosas no sinal capazes de alterar significativamente as medições. Também era relevante que o acesso à informação fosse possível somente para pessoas autorizadas, envolvendo conceitos como autenticação e codificação, para proteger os dados coletados quando sigilosos.

SIMIOS é um sistema que, estruturalmente, poderia ser contextualizado em praticamente qualquer aplicação que se tenha algo a ser rastreado, seja um ser vivo ou não, para qual o GPS seja impraticável ou tenha precisão insuficiente. Portanto, de maneira geral, também era interessante que o sistema tivesse escalabilidade em todos os aspectos - que os dispositivos embarcados nos animais possuíssem sensores diversos e que, possivelmente, toda a aplicação suportasse que uma quantidade maior de variáveis e de usuários fosse inserida.

\begin{figure}[ht]
  \centering
  \caption{Resumo gráfico do sistema}
    \includegraphics[scale=0.7]{esquematico}
  \centerline{\small{Fonte: autores}}
\end{figure}
\FloatBarrier
