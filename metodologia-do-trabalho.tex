\chapter{Metodologia do Trabalho}
Seguindo as orientações de tópicos aplicados a projetos de sistemas embarcados indicadas pelo professor orientador \cite{apostilaSE}, inicialmente foi necessário levantar e descrever os requisitos funcionais e não funcionais do sistema - o que é esclarecido no capítulo seguinte.

A segunda etapa consistia na definição da arquitetura do projeto, para qual os possíveis componentes são associados estabelecendo uma composição que cubra os requisitos funcionais do projeto. A arquitetura é descrita no capítulo 6.

A seguir, aos componentes visíveis no plano físico, o que é feito no capítulo 5. Por fim, é implementada a arquitetura que foi projetada.

Foi adotada metodologia \emph{top-down} visto que primeiro é definido o produto que se deseja obter para poder segmentar as tarefas a serem realizadas em microsserviços.

Ao mesmo tempo o projeto utilizava duas abordagens. Em alguns aspectos é utilizado o modelo em cascata, pois, por se tratar de um trabalho de formatura, é natural que algumas otimizações sejam reservadas para projetos futuros. Por esse ponto de vista, é como se cada projeto fosse uma curva na espiral.

Por outro lado, dada a abrangência tecnológica do projeto, em frentes como a interface será adotado o modelo espiral, pois neste caso modificações eventuais são menos custosas em tempo e orçamento.
