\chapter{Considerações Finais}

\section{Conclusões do Projeto de Formatura}
O presente projeto é composto por módulos bastante distintos e elaborados em diversos níveis de abstração: apresentando considerações acerca do hardware e firmware de placas a serem utilizadas até a elaboração do software de uma aplicação web de alto nível. Portanto ostenta intrincada configuração e integração de seus componentes resultando em uma gama de partes móveis que não são imunes a erros, apesar da cautela dos projetistas.

Contudo, todos os objetivos principais propostos que formam o cerne da funcionalidade: de obter informações acerca do animais de forma constante, de armazenar essas informações para partes interessadas e de apresentá-las de forma clara; foram atendidos pela presente implementação. O primeiro se correspondeu pela interação entre o módulo aplicado aos macacos e aquele aplicado às árvores que mantém contínua vigília sobre os animais enviando informações sobre sua localização. O segundo se alcançou na construção de um banco de dados contendo todos os dados observados de onde o sistema for instalado, cujo conteúdo disponibilizar-se-ia a quaisquer pesquisadores interessados para análise dos dados. O último se concretiza na disponibilidade dos dados em forma de mapas, gráficos ou tabelas, como for mais proveitoso, através de uma aplicação web de fácil utilização e que segue parâmetros de Experiência de Usuário como descrito por Preece, Sharp e Rogers em [12].

Em relação aos requisitos não funcionais levantados, destaca-se o cumprimento da exigência de generalização em relação ao objeto de observação do sistema; de fato, apesar de destinado à aplicação em símios de pequeno porte, com pouca ou nenhuma alteração o sistema pode ser facilmente adaptável para qualquer outro animal de comportamento similar ou a objetos inanimados que se queira investigar. Isto se deve à confirmação de que toda a base do sistema até o nível do banco de dados não exige configuração de qualquer parâmetro relacionado aos animais ou ao seu ambiente. Conclui-se então que o sistema é genérico o bastante para diversas aplicações, mas ressalva-se a necessidade de adaptar as informações coletadas para o novo domínio do problema.

Por outro lado, alguns aspectos do trabalho ficaram aquém dos resultados esperados. Como exemplo pode-se expor o fato dos módulos a serem carregados pelos macacos não atenderem à especificação de peso determinada, constituindo mais do que 10g, impedindo a observação de símios de menor porte. Isso não inviabiliza a monitoração pretendida, visto que a posição dos macacos ainda é assimilada e a monitoração da maioria dos animais não se torna irrealizável.

Outro exemplo pode ser feito da escolha de utilizar um nó central para condensar as informações do módulos dos pontos de acesso no coletor que, em contrapartida, prejudica a escalabilidade do sistema já que demanda que todos os pontos de acesso estejam a distância de transmissão do nó central. É possível resolver este ponto adaptando o sistema para que seja usado um protocolo de Redes Mesh, viabilizando a expansão da rede.

Além disso, em sua presente iteração o sistema não usufrui da acoplação de outros sensores embutidos no SensorTag como foi planejado, já que o foco residia na aquisição de informações sobre o posicionamento. Adaptações em todos os âmbitos do projeto devem ser feitas para possibilitar a captação e exibição de dados de outros sensores presentes na placa.

Dois pontos finais devem ser notados neste balanço. Primeiramente a dificuldade de instalação do sistema se mostrou uma questão complexa e imprevista em sua concepção. Tanto pela variação ambiental da constante necessária para calcular as distâncias a partir do RSSI de forma confiável, como pelas interferências no sinal e limites de distância aos quais a comunicação entre as placas estão sujeitas mostra-se que os requerimentos para o bom funcionamento do SIMIOS não são triviais. Isso apresenta dificuldade notável na implementação do sistema em reservas mais diversas ou próximas a ambientes urbanos, como é o caso do Instituto Butantan. Para tanto um estudo mais aprofundado da variação das constantes ao longo do espaço disponível e uma redundância maior na transmissão dos sinais seriam benéficos para o projeto.

Finalmente, não se deve ignorar os custos financeiros relacionados à possível implantação deste sistema em grande escala. É necessário considerar que para cada objeto de estudo é necessário adquirir uma placa SensorTag, assim como para cada incremento na área de observação, além disso há custo marginais na ampliação do banco de dados e servidores, o que resulta em um gasto nada trivial na expansão do sistema. Não obstante, dadas as alternativas de tecnologias atualmente disponíveis, apresentadas no segundo capítulo deste documento, a implementação proposta ainda se revela a opção de menor custo orçamentário.

\section{Contribuições}

Com o cumprimeto das funcionalidades necessárias neste sistema, acredita-se que o mesmo há de impactar de forma positiva os trabalhos de pesquisa em reservas nas quais se venha a instalar. Foi com o benefício de profissionais acadêmicos e de saúde animal que se baseou a elaboração das linhas guia deste projeto.

Em virtude dos estudos sobre as tecnologias presentes e das informações obtidas diretamente de profissionais da área, é possível dizer que, em comparação com as opções disponíveis de monitoramento de animais no mercado, as funcionalidades do SIMIOS atendem mais veementemente as necessidades correntes de primatologistas. Portanto o sistema contribui com uma solução mais bem adaptada aos profissionais de saúde e de comportamentalismo animal dadas as necessidades apresentadas destes grupos.

Atualmente o potencial de pesquisa nos campos de estudo da monitoração de animais, tanto no que diz respeito a pesquisas biológicas, como veterinárias quanto na esfera da Saúde Única é pouco explorado. Logo também é de interesse dos autores, não só satisfazer os requisitos do presente projeto, mas também dar destaque para tal nicho. Espera-se então que este trabalho sirva como inspiração e/ou ponto de partida para outros que almejam objetivos próximos através de metodologias similares. Adiciona-se a isso a expectativa de contribuições futuras como alternativas ao próprio SIMIOS, seguindo ou não os mesmos paradigmas ou as mesmas metodologias.

\section{Perspectivas De Continuidade}

Dadas as expectativas não cumpridas do trabalho, as primeiras propostas de continuidade pretendem atendê-las. Primeiramente a escalabilidade do sistema pode ser melhorada com o uso de outras técnicas de comunicação, como redes Mesh, na transmissão de dados entre módulos periféricos e pontos de acesso, permitindo uma configuração que consiga cobrir uma área maior. No mesmo âmbito, os sensores da placa podem ser configurados para enviar seus dados coletados juntamente com as mensagens sobre distância, com suporte nos módulos de envio e armazenamento é possível coletar uma gama maior de informações, limitada apenas à variedade de sensores na placa e ao tamanho máximo das mensagens transmitidas. Isso permitiria, por exemplo, a obtenção de informações sobre a temperatura de animais na região, uma das principais métricas no mapeamento da disseminação da febre amarela.

Para outros projetos que pretendem aproveitar os resultados deste, facilmente é possível adaptar o sistema para demais ambientes de uso, onde, além de problemas peculiares de cada ambiente, deve-se apenas atentar à determinação da constante eletromagnética do meio em questão e a eventuais desafios na transmissão de dados, como interferências comuns em comunicações sem fio. Além disso, a adaptação é igualmente fácil para outros objetos de estudo a serem observados, notando-se apenas que a mudança do domínio do problema pode acarretar mudanças em seus requisitos, necessitando, por exemplo, de continentes mais apropriados para as placas SensorTags considerando outras espécies de animais.

Dentre outros fatores de melhoria viáveis é possível destacar: o aumento do paralelismo e uso de threads tanto nos módulos de ponto de acesso, para garantir que mais leituras possam ser feitas em um mesmo local, quanto no módulo coletor, para evitar que este se torne um gargalo da aplicação; a realização de mais transmissões nas comunicações ou a exploração de mais componentes no protocolo para garantir a integridade das informações transmitidas; e a realização de mais medições de um mesmo módulo periférico em um curto espaço de tempo, cuja combinação dos valores permitiria obter resultados mais precisos sem a grandes perdas em termos de eficiência energética.
Por fim, se o compromisso entre fatores for reavaliado é possível dar sequência ao projeto utilizando outra tecnologia na obtenção da distância, como o uso de placas mais baratas, mas com menor variedade de sensores ou protocolos de comunicação. Dessa forma seria possível trocar a variedade de informações coletadas pela redução no custo. Ou contrariamente, pode-se adquirir placas mais caras, porém com alcance superior ou faixa de transmissão menos suscetível a falhas. Trocando neste caso um aumento no custo financeiro do projeto por uma precisão e confiabilidade maior dos dados. De qualquer forma que se pense refatorar o sistema, partes dele, como o banco de dados ou a aplicação web nos casos exemplificados, ainda podem ser, e recomenda-se que sejam, reaproveitadas.
