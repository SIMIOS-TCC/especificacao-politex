\chapter{Tecnologias Utilizadas}
Considerando os aspectos discutidos no capítulo 2 sobre possíveis e impossíveis instrumentos para o nosso contexto, por fim foram selecionadas as tecnologias que serão efetivamente utilizados no projeto, as quais são descritas neste capítulo.

\section{Dispositivos Embarcados}

Partindo-se deste requisito, selecionou-se a plataforma de desenvolvimento \emph{SensorTag} da \emph{Texas Instruments} como componente embarcado que ficará em cada macaco e nos pontos de acessos nas árvores. Trata-se de uma placa leve e comercialmente atraente, que contém 6 sensores, incluindo de temperatura, e Bluetooth de Baixo Consumo Energético (BLE - \emph{Bluetooth Low Energy}) ou Sub-1GHz. Seu \emph{datasheet} pode ser encontrado nas referências deste trabalho \cite{datasheet}.

Além do \emph{SensorTag}, foi selecionado um \emph{Raspberry Pi} 2 para receber os dados de uma placa \emph{SensorTag} central via Transmissor/Receptor Assíncrono Universal (UART - \emph{Universal Asynchronous Receiver/Transmitter}). Tal placa central deverá receber dados dos demais pontos de acesso, e estes dados serão armazenados em um servidor pelo \emph{Raspberry Pi}.

\subsection{Sensortag}

\begin{figure}[ht]
  \centering
  \caption{Componentes do \emph{SensorTag}}
    \includegraphics[scale=0.5]{sensortag}
  \centerline{\small{Fonte: \emph{Texas Instruments}}}
\end{figure}

O código da placa dentre os produtos da TI é CC1350STK. Este é um dispositivo que é facilmente integrável com aplicações de tecnologia móvel (tendo inclusive um aplicativo aberto oficial da TI que pode ser baixado nas lojas de aplicativo e ser usado em conjunto com as placas).\\

A placa contém um microcontrolador que tem funcionalidades de comunicação sem fio, um processador principal \emph{ARM Cortex} M3 de 32 \emph{bits} com \emph{clock} de 48MHz e uma variedade de periféricos, como temporizadores de propósito geral, conversor analógico-digital, pinos de propósito geral, entre outros.

\section{Servidor}
Foi escolhido o banco de dados relacional \emph{MySQL} da \emph{Oracle} por se tratar de um sistema de código aberto simples, embora completo.

O projeto SIMIOS prevê o rastreio de animais em reservas, o que envolve, como visto anteriormente, no máximo cerca de 10 animais em grandes reservas. Dessa forma, sabemos que não envolve sobrecarga de acessos por segundo e mesmo isto poderia ser corrigido com \emph{buffering}.

Um ponto negativo do \emph{MySQL} é que sua escalabilidade pode ser prejudicada - cada servidor tem um tamanho limitado e cada conjunto de dados só pode ser alocado em um servidor (não suporta particionamento), o que pode ser prejudicial em casos que deseja-se guardar no banco grande quantidade de dados. Para corrigir tal empecilho é possível implementar importação de dados.

\section{Software}
Para desenvolvimento do programa computacional que é executado no servidor, foi escolhida a linguagem de programação \emph{Java}, com suporte de \emph{frameworks} \emph{Spring} e \emph{Java Persistence API} (JPA), facilitando principalmente os processos de queries do banco de dados, de autenticação e autorização e de mapeamento de interface \emph{model-view-controller} (MVC).

Por se tratar de uma aplicação de histórico de dados, pouquíssimo processamento está previsto e a computação pode ser realizada em tempo real pelo computador do usuário. Dessa forma, não se faz necessário o uso de linguagem de programação de execução eficiente (C++, por exemplo).
