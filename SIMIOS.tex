\documentclass[]{politex}
% ========== Opções ==========
% pnumromarab - Numeração de páginas usando algarismos romanos na parte pré-textual e arábicos na parte textual
% abnttoc - Forçar paginação no sumário conforme ABNT (inclui "p." na frente das páginas)
% normalnum - Numeração contínua de figuras e tabelas
%	(caso contrário, a numeração é reiniciada a cada capítulo)
% draftprint - Ajusta as margens para impressão de rascunhos
%	(reduz a margem interna)
% twosideprint - Ajusta as margens para impressão frente e verso
% capsec - Forçar letras maiúsculas no título das seções
% espacosimples - Documento usando espaçamento simples
% espacoduplo - Documento usando espaçamento duplo
%	(o padrão é usar espaçamento 1.5)
% times - Tenta usar a fonte Times New Roman para o corpo do texto
% noindentfirst - Não indenta o primeiro parágrafo dos capítulos/seções


% ========== Packages ==========
\usepackage[utf8]{inputenc}
\usepackage{amsmath,amsthm,amsfonts,amssymb}
\usepackage{graphicx,cite,enumerate}
\usepackage{makecell}
\graphicspath{ {./images/} }
\usepackage{placeins}
\usepackage{enumitem}
\usepackage{listings}
\usepackage{color}
\usepackage[final]{pdfpages}

\definecolor{dkgreen}{rgb}{0,0.6,0}
\definecolor{gray}{rgb}{0.5,0.5,0.5}
\definecolor{mauve}{rgb}{0.58,0,0.82}

\lstset{frame=tb,
  language=C,
  aboveskip=3mm,
  belowskip=3mm,
  showstringspaces=false,
  columns=flexible,
  basicstyle={\small\ttfamily},
  numbers=none,
  numberstyle=\tiny\color{gray},
  keywordstyle=\color{blue},
  commentstyle=\color{dkgreen},
  stringstyle=\color{mauve},
  breaklines=true,
  breakatwhitespace=true,
  tabsize=3
}


% ========== Language options ==========
\usepackage[brazil]{babel}
%\usepackage[english]{babel}


% ========== ABNT (requer ABNTeX 2) ==========
%	http://www.ctan.org/tex-archive/macros/latex/contrib/abntex2
\usepackage[num]{abntex2cite}

% Forçar o abntex2 a usar [ ] nas referências ao invés de ( )
\citebrackets{[}{]}


% ========== Lorem ipsum ==========
\usepackage{blindtext}



% ========== Opções do documento ==========
% Título
\titulo{SIMIOS - Sistema de Monitoramento de Interação Orgânica entre Símios}

% Autor
\autor{Larissa Mangolim Amaral \\%
		Luciana da Costa Marques \\%
		Pedro Oscar Gallo Vaz}

% Para múltiplos autores (TCC)
%\autor{Nome Sobrenome\\%
%		Nome Sobrenome\\%
%		Nome Sobrenome}

% Orientador / Coorientador
\orientador{Prof. Dr. Carlos Eduardo Cugnasca}
\coorientador{Prof. Dr. Bruno de Carvalho Albertini}

% Tipo de documento
\tcc{Eletricista com ênfase em Computação}
%\dissertacao{Engenharia Elétrica}
%\teseDOC{Engenharia Elétrica}
%\teseLD
%\memorialLD

% Departamento e área de concentração
\departamento{Departamento de Engenharia de Computação e Sistemas Digitais}
\areaConcentracao{Engenharia Elétrica}

% Local
\local{São Paulo}

% Ano
\data{2018}




\begin{document}
% ========== Capa e folhas de rosto ==========
\capa
\falsafolhaderosto
\folhaderosto

%\includepdf{EPUSP-Catalogacao-na-Fonte}

% ========== Folha de assinaturas (opcional) ==========
\begin{folhadeaprovacao}
	\assinatura{Prof.\ Livre-Docente Carlos Eduardo Cugnasca}
%	\assinatura{Prof.\ Y}
%	\assinatura{Prof.\ Z}
\end{folhadeaprovacao}


% ========== Ficha catalográfica ==========
% Fazer solicitação no site:
%	http://www.poli.usp.br/en/bibliotecas/servicos/catalogacao-na-publfile:///C:/Users/laris/Desktop/poliTeX-master/aspectos-conceituais.texicacao.html

\includepdf[pages=-,pagecommand={},width=\textwidth]{EPUSP-Catalogacao-na-Fonte.pdf}

% ========== Dedicatória (opcional) ==========
%\dedicatoria{Dedicatória}


% ========== Agradecimentos ==========
\begin{agradecimentos}
Às nossas famílias e amigos, por todo o apoio, suporte e carinho imensuráveis.

Aos professores orientadores, agradecemos a dedicação para o auxílio intelectual e encaminhamento deste trabalho.

À Professora Doutora Cristiane Schilbach Pizzutto, por sua saborosa contribuição para nossa compreensão da óptica do cientista veterinário.

Ao professor Reginaldo Arakaki, por ter disponibilizado os recursos iniciais para a compra das placas da Texas Instruments e com isso ter permitido os primeiros passos do projeto. 

À professora Lúcia Filgueiras, por ter dado uma visão multidisciplinar do que se poderia fazer no aprimoramento da Interface Homem-Computador.

Ao professor Jorge Kinoshita, pelo apoio semanal nas aulas da disciplina de Laboratório de Sistemas Embarcados, pelas críticas construtivas e preocupação sincera com o sucesso do trabalho.

Ao pós-doutorando Wilian França Costa, pela orientação para a familiarização das ferramentas do projeto durante todo o ano.

Aos demais professores e colegas que contribuíram através de nossa formação como engenheiros e como pessoas.
\end{agradecimentos}

% ========== Epígrafe (opcional) ==========
\epigrafe{
	\emph{``Evidentemente, para os que não têm consciência do significado das heranças paisagísticas e ecológicas, os esforços dos cientistas que pretendem responsabilizar todos e cada um pela boa conservação e pelo uso racional da paisagem e dos recursos da natureza somente podem ser tomados como motivo de irritação, quando não de ameaça, a curto prazo, à economicidade das forças de produção econômica.''}
	\begin{flushright}
		-{}- Aziz Nacib Ab'Sáber
	\end{flushright}
}


% ========== Resumo ==========
\begin{resumo}

A execução deste trabalho visa a projeção e implementação de um sistema capaz de auxiliar pesquisadores de saúde e biologia animal no monitoramento comportamental de uma população de animais em estudo dentro de reservas. Para tal, foi elaborada uma arquitetura composta por uma rede de sensores, um banco de dados e uma aplicação \emph{web}. As informações sobre a localização dos animais são obtidas pela rede de sensores e armazenadas no banco de dados, onde são acessadas pela aplicação \emph{web} para serem analisadas por pesquisadores. O desenvolvimento deste sistema tem como finalidade atender às necessidades de monitoramento de símios na reserva natural do Instituto Butantan da Universidade de São Paulo. Contudo, pretende-se que sua aplicação seja possível em distintas reservas, para a observação de diferentes animais e que possa eventualmente ser expandido para captar demais variáveis de controle relevantes para o biólogo ou veterinário. Finalmente, o sistema obtido cumpre os requisitos definidos e demonstra escalabilidade.

\textbf{Palavras-Chave} -- Monitoramento animal remoto, Redes de sensores sem fio, Sistema embarcado.
\end{resumo}


% ========== Abstract ==========
\begin{abstract}
This study presents the design and implementation of an entire system capable of assisting animal health and biology researchers monitoring the object of study. Therefore, an embedded wearable was programmed so that the system could provide the researcher remotely with the observed animals’ location. For that to happen, they would have to be living in a natural reserve mapped by a wireless sensor network. This system’s purpose is to meet Instituto Butantã’s needs regarding monitoring simios inside the reserve. However, it is expected that this system is also applicable inside many different reserves, for distinct mammals observation and open to be expanded so that it shall be able to monitor as many other control variables the biologist finds relevant. The final work meets the defined requirements and shows scalability.

\textbf{Keywords} -- Remote animal monitoring, Wireless sensor network, Embedded.
\end{abstract}

% ========== Listas (opcional) ==========
\listadefiguras
\listadetabelas

% ========== Listas definidas pelo usuário (opcional) ==========
\begin{pretextualsection}{Lista de abreviaturas e siglas}

AP \quad \quad $\>$ \textit{Access Point}

CRUD \quad \textit{Create Read Update Delete}

GPS \quad \quad \textit{Global Positioning System}

ID \quad \quad \quad Identificador

MCU \quad $\>$ \textit{Microcontroller Unit}

MVC \quad $\>$ \textit{Model-view-controller}

PWM \quad $\>$ \textit{Pulse Width Modulation}

RFID \quad $\>$ \textit{Radio-Frequency Identification}

RSSI \quad \quad \textit{Received Signal Strength Indication}

RSSF \quad \quad Redes de Sensores Sem Fio

SQL \quad \quad $\>$ \textit{Structured Query Language}

TI \quad \quad \quad $\>$ \textit{Texas Instruments}

UART \quad $\>$ \textit{Universal Asynchronous Receiver/Transmitter}

UX \quad \quad \quad \textit{User Experience}

\end{pretextualsection}

% ========== Sumário ==========
\sumario



% ========== Elementos textuais ==========

\chapter{Introdução}
	
\section{Objetivo}
Este trabalho visa o projeto e implementação de um sistema capaz de obter a posição relativa de macacos em reservas, dentre outros dados do ambiente ou do animal.

A composição do sistema prevê (1) dispositivos embarcados inseridos em mochilinhas anexadas ao macaco, que, em conjunto, compõem (2) uma rede de sensores para adquirir as informações necessárias do ambiente e enviá-las para (3) um servidor. Este realiza o processamento e armazenamento dos dados que serão injetados em (4) uma interface em software disponível para o usuário.

A discriminação do sistema é melhor realizada nos capítulos 4 e 6.



\chapter{Aspectos Conceituais}
Alguns dos principais conceitos para compreensão e contextualização deste projeto são trabalhados neste capítulo.
	
\section{Macacos e Reservas Naturais}
Para melhor compreensão do aspecto biológico que este trabalho toca, foi realizada uma entrevista com a professora Cristiane Pizzutto, que pode ser vista na íntegra no apêndice 1 deste documento.

A partir desta, foi possível quantificar alguns parâmetros importantes para o dimensionamento do projeto, tal como a quantidade comumente observada de animais em bandos de reservas e cativeiros, para qual foi assegurado que, considerando um bando de dez macacos, estaríamos abrangendo seguramente o suficiente.

Também foi possível notar como a tarefa de observação para aquisição de dados relativos aos animais, tais como sinais vitais, movimentação e alimentação, é exaustiva, toma tempo e pode ser objetiva o bastante para que possa ser realizada por um sistema remoto.

A parte subjetiva do levantamento de dados está relacionada às atividades e interações dos animais, que normalmente só podem ser adquiridos por observação direta. Essa é a parte que nosso projeto tenta abordar e que nos fez perceber que talvez, para cativeiro, o auxílio de câmeras com alguma inteligência seria bem vindo.

Outra questão levantada é de como estes dados são digeridos. A pesquisadora aponta que a manipulação dos dados é manual e que utilizam planilhas para obter estatísticas. Considerando a quantidade massiva de dados, seria interessante pensar em injetar conceitos de Big Data.

\section{Tecnologias Potenciais}
\textbf{Redes de Sensores Sem Fio}

A emergência da tecnologia de Redes de Sensores Sem Fio (RSSF) permitiu não somente o monitoramento das variáveis de um objeto, mas também o supervisionamento de todo o contexto em que ele está incluído e da interação dele com os demais pontos do sistema sendo sensoriados.

RSSFs são especialmente relevantes quando se tratando de ambientes cuja área que deve ser coberta é muito extensa. Estas redes são compostas por nós interligados, em que cada nó deve ter sensores, processamento, memória, antena e bateria independentes e sustentáveis.

Dada essa composição, as RSSFs são capazes de satisfazer áreas de cobertura muito extensas e são ótimas para criar integração entre elementos que estão, localmente, constantemente conjuntos.

Por isso, compõem uma tecnologia ótima para organizações biológicas, que envolvem grandes populações distribuídas, sendo, portanto, frequentemente aplicadas em sistemas agrícolas.

Como enfatizado por Handcock (2009), para animais essa utilidade também é incluída, mas prevê algumas ressalvas. Uma delas considera a situação de que o animal de vida livre pode permanecer por semanas fora do alcance de pontos de acesso que recebam seus dados e, por esse motivo, cada nó da rede deverá possuir bateria e memória suficientemente robustas. Para reduzir o tempo de ausência de resposta de um determinado indivíduo, é possível implementar escuta nos próprios nós da rede, de forma que os animais que entrarem em contato entre si mantenham as informações dos demais, aumentando a chance de que algum deles possa transmiti-las para o servidor no alcance de pontos de acesso. Essa prática, no entanto, exige ainda mais energia e armazenamento.

RSSFs são, por vezes, estudadas para posicionamento em ambientes internos, principalmente atreladas ao uso de pontos de acesso Wi-fi, que já são naturalmente alocados nesses espaços para uso de internet.

\textbf{Bluetooth Low Energy (BLE)}

O Bluetooth Low Energy vem como alternativa ao Wi-fi, com menor alcance mas o superando pelo baixo consumo energético, especialmente para aplicações em locais nos quais a disponibilidade energética é limitada e inconstante. Isso é equilibrado por otimizações na arquitetura de dispositivos com BLE que, de certa forma, reduzirá sua eficiência. Um exemplo desse tipo de otimização é o funcionamento do relógio que, na maioria dos dispositivos, possui hardware com resolução elevada.

Como é enfatizado por Amaral e Biscaro (2017), o BLE também possui baixo custo e disseminação eminente, com o apoio da Apple no lançamento do IBeacon e integração mobile.

Este projeto de fato é ambientado em local com indisponibilidade energética e, ao mesmo tempo, não se é requisitado alto processamento. Assim, foi escolhido trabalhar com o Bluetooth e, por isso, sua arquitetura será explanada brevemente.

O BLE possui duas estruturas principais: o GAP (Generic Access Profile) e o GATT (Generic Attribute Profile). O GAP é responsável pela visibilidade dos dispositivos e pelo controle das conexões entre os mesmos. Para isso, os dispositivos são categorizados como periférico (dispositivos menores que consomem menos energia e processamento e se conectam ao dispositivo central) ou central.

Os dispositivos periféricos se anunciam para tornarem-se visíveis através de pequenos pacotes que são enviados periodicamente em broadcast. Tais pacotes são denominados Advertising Data e constam de até 31 bytes. Os dispositivos centrais são responsáveis pela escuta. Recebem os pacotes anunciados pelos periféricos e identificam se desejam realizar conexão e respondem o respectivo dispositivo. Uma vez que a conexão é estabelecida, o broadcast é normalmente interrompido e é estabelecido um serviço GATT, concluindo que um periférico só possa se conectar a um dispositivo central de cada vez.

O GATT, por sua vez, define a forma como dois dispositivos BLE transferem dados através de uma conexão dedicada bidirecional entre eles. Para isso, haverá uma topologia mestre (GATT Client - que faz o request) escravo (GATT Server - que faz o response).

\textbf{Rádio}
A aplicação tecnológica para rastreamento de animais mostra-se limitada a transmissores de rádio por muitos anos, por mais que a tecnologia de localização para aplicações humanas já o tenha superado de longe.

O RFID foi bastante usado para obter informações relacionadas à condição do animal identificado. Recentemente, estes transmissores têm sido também utilizados para detectar encontros sociais entre animais através de picos de intensidade do sinal de rádio sendo transmitido.

\textbf{GPS}
A emancipação do GPS aplicado ao smartphone praticamente trivializou a tarefa de localização, principalmente quando associada à mobilidade e roteamento.

A integração de tal tecnologia em sistemas biológicos demonstra uma tentativa de integrar o sensoriamento da interação do animal com o ambiente, como é salientado por Handcock [5].

O GPS, no entanto, tem uma série de complicações. Primeiramente, sua precisão em baixa escala é bem complicada. Handcock afirma que para obter boa acurácia, é necessário ter uma taxa de amostragem relativamente alta, o que é bastante ruim para a sustentabilidade da memória e da energia do sistema.

Um outro problema está relacionado à pouca praticidade do módulo GPS, que apresenta peso relativamente elevado (aproximadamente 10g) dependendo do animal que está sendo rastreado.

\section{Modelos Comerciais}
Alguns modelos de coleiras voltadas a mamíferos são citadas na tabela a seguir. Os produtos são fornecidos pela empresa ATS que, infelizmente, não discrimina o porte recomendado do animal usuário em seu site, portanto só foi possível inferir o peso que o dispositivo deste projeto deveria ter confirmando o que havia sido relatado pelos pesquisadores: de algo de no máximo 10g, uma vez que macacos menores pesam cerca de 400g.

Por outro lado, foi possível conceber alguns potenciais modelos para o invólucro do produto final deste projeto, principalmente no que diz respeito ao material utilizado.

Como visto no item anterior, comercialmente é utilizado rádio na maior parte dos casos (M17X0 e M15X5) e, eventualmente, GPS (W500 - para o qual é possível notar que exige um peso bem superior ao limite estabelecido de cerca de 10g).

\begin{table}[h]
\centering
\caption{Exemplos comerciais de coleiras de rastreamento de mamíferos da ATS}
\vspace{0.5cm}
\begin{tabular}{l|ccc}
\hline
Nome & \makecell{SM17X0 Mammal \\ Collar, X-Small} & \makecell{M15X5 Mammal \\ Zip-Tie Collar} & \makecell{W500 Wildlink GPS \\ Logger, Small Collar} \\
Imagem & \includegraphics[scale=0.5]{ATS1} & \includegraphics[scale=0.5]{ATS2} & \includegraphics[scale=0.5]{ATS3} \vspace{0.4cm}\\

Peso & 9 a 16g & 10 a 40g & 65 a 115g \vspace{0.4cm}\\

Bateria & Lítio / 156 a 282 dias & Lítio / 195 a 596 dias & AA / 1,75 a 3,5 anos \vspace{0.4cm} \\

Material & 
\makecell{- Coleira de \\ \textbf{neoprene} \\
- Encapsulamento \\ de resina a prova \\ de água} &
\makecell{ - Coleira de \textbf{tubo} \\ \textbf{de plástico (cable-tie)} \\
- Encapsulamento \\ de resina a prova \\ de água} &
\makecell{- Coleira de \textbf{neoprene} \\ \textbf{ou nylon} }   
\end{tabular}
\end{table}


\chapter{Metodologia do Trabalho}
Seguindo as orientações de tópicos aplicados a projetos de sistemas embarcados indicadas pelo professor orientador \cite{apostilaSE}, inicialmente foi necessário levantar e descrever os requisitos funcionais e não funcionais do sistema - o que é esclarecido no capítulo seguinte.

A segunda etapa consistia na definição da arquitetura do projeto, para qual os possíveis componentes são associados estabelecendo uma composição que cubra os requisitos funcionais do projeto. A arquitetura é descrita no capítulo 6.

A seguir, aos componentes visíveis no plano físico, o que é feito no capítulo 5. Por fim, é implementada a arquitetura que foi projetada.

Foi adotada metodologia \emph{top-down} visto que primeiro é definido o produto que se deseja obter para poder segmentar as tarefas a serem realizadas em microsserviços.

Ao mesmo tempo o projeto utilizava duas abordagens. Em alguns aspectos é utilizado o modelo em cascata, pois, por se tratar de um trabalho de formatura, é natural que algumas otimizações sejam reservadas para projetos futuros. Por esse ponto de vista, é como se cada projeto fosse uma curva na espiral.

Por outro lado, dada a abrangência tecnológica do projeto, em frentes como a interface será adotado o modelo espiral, pois neste caso modificações eventuais são menos custosas em tempo e orçamento.


\chapter{Especificação de Requisitos de Sistema}
O funcionamento essencial do sistema, o que define seus requisitos funcionais, requeriu que a posição dos macacos seja possível de ser medida, armazenada e mostrada para o usuário.

Além desses, foram levantados os requisitos não funcionais, que trabalham aspectos necessários e complementares para o bom funcionamento do sistema, muitas vezes previstos pelo público solicitante do mesmo.

No SIMIOS, os principais requisitos não funcionais foram apontados por pesquisadores biólogos e veterinários com experiência em monitoramento de macacos. Dentre eles, está que o peso da mochila que será anexada ao animal não deveria ultrapassar 10g para não influenciar em seu comportamento nem sobrecarregá-lo, visto que o principal grupo de foco (saguis) tem peso médio de 400g. Para isso, era interessante que todos os componentes da mochila fossem o mais leves possível.

Outra situação apontada é o fato de que toda vez que a bateria do aparelho tiver de ser trocada, o veterinário deverá capturar o macaco e sedá-lo, o que é bastante prejudicial para a confiança que o animal constrói pelo ser humano. Dessa forma, era desejável que a eficiência energética do dispositivo embarcado seja alta para que a bateria tenha de ser trocada com a menor frequência possível.

Além da mochila do animal, os dados coletados deveriam ser confiáveis. Isso envolve garantir a validade das medidas enviadas e a minimização de seus erros. Assim, remedia-se casos nos quais há medidas falsas, por exemplo, quando o dispositivo é removido acidentalmente do animal, bem como nos quais haja interferências ruidosas no sinal capazes de alterar significativamente as medições. Também era relevante que o acesso à informação fosse possível somente para pessoas autorizadas, envolvendo conceitos como autenticação e codificação, para proteger os dados coletados quando sigilosos.

SIMIOS é um sistema que, estruturalmente, poderia ser contextualizado em praticamente qualquer aplicação que se tenha algo a ser rastreado, seja um ser vivo ou não, para qual o GPS seja impraticável ou tenha precisão insuficiente. Portanto, de maneira geral, também era interessante que o sistema tivesse escalabilidade em todos os aspectos - que os dispositivos embarcados nos animais possuíssem sensores diversos e que, possivelmente, toda a aplicação suportasse que uma quantidade maior de variáveis e de usuários fosse inserida.

\begin{figure}[ht]
  \centering
  \caption{Resumo gráfico do sistema}
    \includegraphics[scale=0.7]{esquematico}
  \centerline{\small{Fonte: autores}}
\end{figure}
\FloatBarrier


\chapter{Tecnologias Utilizadas}
Considerando os aspectos discutidos no capítulo 2 sobre possíveis e impossíveis instrumentos para o nosso contexto, por fim foram selecionadas as tecnologias que serão efetivamente utilizados no projeto, as quais são descritas neste capítulo.

\section{Dispositivos Embarcados}
Partindo disso, selecionou-se a plataforma de desenvolvimento do Sensor Tag da Texas Instruments como componente embarcado de cada macaco. Trata-se de uma placa leve que contém 6 sensores, incluindo de temperatura, e comunicador BLE. Seu datasheet pode ser encontrado nas referências deste trabalho.

De início, foi selecionado o protocolo de comunicação BLE, especialmente por se tratar de um sistema que requer absoluto cuidado com o consumo energético, como enfatizado no capítulo anterior.
Partindo-se deste requisito, selecionou-se a plataforma de desenvolvimento Sensor Tag da Texas Instruments como componente embarcado que ficará em cada macaco e nos pontos de acessos nas árvores. Trata-se de uma placa leve e comercialmente atraente, que contém 6 sensores, incluindo de temperatura, e comunicação por BLE. Seu datasheet pode ser encontrado nas referências deste trabalho [1].

Além do Sensortag, foi selecionado um Raspberry Pi 2 para receber os dados de uma placa sensortag central via UART. Tal placa central deverá receber dados dos demais pontos de acesso, e estes dados serão armazenados em um servidor pelo Raspberry Pi.

\subsection{Sensortag}

\begin{figure}[ht]
  \centering
    \includegraphics[scale=0.5]{sensortag}
  \caption{Componentes do Sensor Tag (Fonte: extraído do site da Texas Instruments)}
\end{figure}

O código da placa dentre os produtos da TI é CC1350STK. Este é um device que é facilmente integrável com aplicações mobile (tendo inclusive um aplicativo aberto oficial da TI que pode ser baixado nas lojas de aplicativo e ser usado em conjunto com as placas).\\

A placa contém um microcontrolador que tem funcionalidades de comunicação wireless, um processador principal ARM Cortex M3 de 32 bits com clock de 48MHz e uma variedade de periféricos, como timers de propósito geral, conversor analógico-digital, GPIOs, entre outros. 

\section{Servidor}
Foi escolhido o banco de dados relacional MySQL da Oracle por se tratar de um sistema open source simples, embora completo.

O projeto SIMIOS prevê o rastreio de animais em reservas, o que envolve, como visto anteriormente, no máximo cerca de 50 animais em grandes reservas. Dessa forma, sabemos que não envolve sobrecarga de acessos por segundo e mesmo isto poderia ser corrigido com buffering.

Um ponto negativo do MySQL é que sua escalabilidade pode ser prejudicada - cada servidor tem um tamanho limitado e cada set de dados só pode ser alocado em um servidor (não suporta particionamento), o que pode ser prejudicial em casos que deseja-se guardar no banco grande quantidade de dados. Para corrigir tal empecilho é possível implementar importação de dados.

\section{Software}
Para desenvolvimento do programa computacional que é executado no servidor, foi escolhida a linguagem de programação Java, com suporte de frameworks Spring e JPA, facilitando principalmente os processos de queries do banco de dados, de autenticação e autorização e de mapeamento de interface model-view-controller (MVC).

Por se tratar de uma aplicação de histórico de dados, pouquíssimo processamento está previsto e a computação pode ser realizada em tempo real pelo computador do usuário. Dessa forma, não se faz necessário o uso de linguagem de programação de execução eficiente (C++, por exemplo).


\chapter{Projeto e Implementação}

Tendo em vista todos os pontos observados anteriormente, é levantado o perfil da arquitetura do sistema, como pode ser visto na figura a seguir. O projeto foi segmentado em camadas de acordo com a tecnologia sendo implementada (linhas pretas pontilhadas).

Também foram designados microsserviços principais que distinguiam o caminho crítico que desejamos obter. Cada um desses foi descrito neste capítulo.

\begin{figure}[ht]
  \centering
    \caption{Diagrama esquemático da arquitetura}
    \includegraphics[scale=0.7]{arquitetura}
  \centerline{\small{Fonte: autores}}
\end{figure}
\FloatBarrier

Todo código elaborado pelos autores com relação a cada projeto é acessível no link da organização do repositório do grupo \cite{github}.

\section{Módulo Do Target}

Essa etapa engloba possibilitar a comunicação unidirecional de cada dispositivo nos animais (\emph{target}). Essa comunicação foi feita através da emissão em \emph{broadcast} de pacotes utilizando Sub-1GHz, que trabalha a frequência de 868MHz, com o protocolo proprietário da TI, chamado \emph{EasyLink Proprietary RF}.

Além disso, era necessário que o \emph{target} tivesse um ciclo de \emph{sleep}, de forma a controlar a taxa de transmissão do sinal, reduzindo a energia consumida por cada dispositivo. Estudos foram feitos para se decidir qual a melhor estratégia de implementação.

\subsection{Comunicação unidirecional}

Para tal, o módulo padrão de transmissão (Tx) disponibilizado pela TI foi estudado e modificado para que o pacote enviado contivesse o Identificador (ID) de cada animal - que ficou codificado de maneira pré definida em cada \emph{SensorTag}.

Esse módulo cria uma \emph{task} para a execução da tarefa de emissão dos pacotes cujo \emph{payload} e endereço de destino podem ser configurados. Como só era desejado o ID do macaco e a intensidade do sinal, o \emph{payload} foi constituído somente por um único \emph{byte} contendo a primeira informação. Tendo isso em vista, é compreensível que seria possível inserir no mesmo \emph{payload} demais informações relativas aos animais através dos sensores em cada placa.

Regular a potência do sinal é interessante uma vez que tanto o alcance do sinal quanto a precisão observada no RSSI variam. Adotamos uma intensidade arbitrária de 12 dB para a qual observamos que para cerca de 30 metros o ponto de acesso ainda recebia resposta.

\subsection{Temporização}

Para controle do ciclo de envio e \emph{sleep} do módulo \emph{target}, foram pesquisadas três principais estratégias:

\begin{enumerate}
  \item Utilização de um periférico do microcontrolador do \emph{SensorTag} do tipo \emph{timer}, para determinação do período de sleep através de seu contador;
  \item Por meio de uma pausa na tarefa do sistema operacional que está executando a transmissão de pacotes;
  \item Ou alterando-se uma constante da estrutura de pacotes.
\end{enumerate}

A \emph{Texas Instruments} permite que o controle dos periféricos dos módulos utilizados seja feito por meio do uso de \emph{API}s específicas de seus \emph{drivers}. No caso do \emph{timer}, é possível utilizar a referência da biblioteca GPTimerCC26XX.h \cite{gptimer}. Ela contém funções básicas de operação do \emph{timer}, como init(), start(), stop() e callback().

Para utilização do \emph{timer}, é necessário também que se configure seus parâmetros. Alguns deles, mais essenciais, são:

\begin{itemize}
	\item Modo (\emph{One-shot}, Modulação por largura de pulso, periódico, dentre outros);
	\item Tamanho máximo do contador (pode ser de 16 ou 32 bits);
	\item Valor até o qual o \emph{timer} deve contar;
	\item Função de retorno;
\end{itemize}

Todos esses parâmetros são referenciados no datasheet do microcontrolador \cite{datasheet} e na referência da biblioteca do \emph{driver} \cite{gptimer}. No contexto do \emph{SensorTag} e deste projeto, o \emph{timer} pode ser configurado para funcionar de maneira periódica e a transmissão de mensagens ser incluída em sua função de \emph{callback}.

O cálculo da periodicidade do \emph{timer} deve ser feita levando-se em conta a sua frequência de atualização, lembrando que, em uma descrição simples, o funcionamento do \emph{timer} consiste em contar de 0 até o valor especificado em seus parâmetros. O memorial de cálculo encontra-se no Apêndice D.

Boa parte dos exemplos de uso de periféricos providos pela TI incluem criação de tarefas. Conforme mencionado no início do capítulo 6, utilizou-se primariamente um dos exemplos fornecidos pela TI para operação do módulo de \emph{target}. Uma solução mais simples do que configurar o \emph{timer} (e possivelmente menos custosa) seria utilizar a função Task\texttt{\char`_}Sleep(nticks) \cite{task-modules}. O que esta função faz em essência é bloquear a tarefa pelo tempo definido em nticks, o qual é medido em microssegundos e tem como referência o próprio \emph{clock} do sistema (48MHz \cite{datasheet}).

Uma última estratégia seria utilizar um dos parâmetros da própria estrutura de mensagens de trasmissão do protocolo utilizado \cite{forum-easylink}. Tal parâmetro é EasyLink\texttt{\char`_}ms\texttt{\char`_}To\texttt{\char`_}RadioTime e é medido em microssegundos, como seu nome sugere \cite{easylink}. Seria uma alternativa às duas últimas mencionadas, porém, há pouca informação sobre seu tamanho máximo e há relatos de outros usuários da API sobre problemas em seu uso por este motivo.

\subsection{Eficiência Energética}

Para efeito de preservação da bateria escolheu-se realizar a transmissão a cada 5 segundos de maneira a manter o aspecto de atualização em tempo real. Tal valor pode ser configurado facilmente alterando-se uma das variáveis contidas nas estratégias de temporização mencionadas anteriormente (loadVal para o \emph{timer}, nTicks para Task\texttt{\char`_}Sleep() e EasyLink\texttt{\char`_}ms\texttt{\char`_}To\texttt{\char`_}RadioTime para a estrutura dos pacotes). Esta mudança deve ser feita caso seja julgada interessante uma maior economia de energia em troca de desempenho.

\section{Módulo Do Ponto De Acesso}

Cada ponto de acesso é responsável pela leitura e armazenamento dos dados correspondentes aos pacotes enviados pelos \emph{targets}. Assim, para cada pacote recebido é possível identificar a intensidade do sinal, que é armazenada em um vetor de tamanho fixo junto com o ID do macaco.

Uma vez que o vetor estiver completo, o ponto de acesso cria e envia pacotes de maneira similar a como é feita pelos \emph{targets}, mas com endereço de destino distinto, por exemplo 0xBB.

Para isso, foi desenvolvido um projeto composto por ambos os módulos padrão de leitura (Rx) e transmissão (Tx). O módulo de leitura é configurado para que passem pelo filtro somente pacotes cujo endereço de destino seja igual ao dos enviados pelo \emph{target}, por exemplo 0xAA.

\section{Módulo Central}

A mediação da comunicação entre o ponto de acesso (\emph{SensorTag}) e o controlador (\emph{Raspberry Pi}) deve ser feita através do nó denominado Central implementado por um \emph{SensorTag} conectado por UART ao controlador.
Este componente terá a função de receber os pacotes filtrados pelo endereço de destino configurado pelos pontos de acesso e repassar as informações por conexão serial para o controlador.

Para isso, analogamente ao realizado nos demais componentes, foi desenvolvido um projeto composto por ambos os módulos padrão de leitura (Rx) e transmissão serial UART disponibilizados pela TI. Essa composição foi feita criando uma \emph{thread} para cada módulo utilizando \emph{buffers} duplos, de forma que cada tarefa trabalhasse paralelamente em um \emph{buffer} individual.

O tamanho dos \emph{buffers} foi dimensionado para abrigar 32 medidas, considerando a memória máxima da placa e, ao mesmo tempo, a não sobrecarregar a comunicação serial com o controlador, que depende da interrupção dessa comunicação para escrever no banco. O módulo UART é configurado para atuar a taxa de transmissão de 115200 bits por segundo sem paridade.

\subsection{Integração dos pontos de acesso e target}

Definidos os módulos de \emph{target}, ponto de acesso e ponto central, o seu funcionamento conjunto é ilustrado conforme o diagrama da Figura 7, conjuntamente com as principais funções de cada módulo.

O projeto Simio-Tx representa o comportamento do módulo \emph{target} que é responsável por emitir pacotes com a identificação do macaco em que está instalado (pacote 1) e possui a função de Task\texttt{\char`\_}Sleep() para diminuir o consumo energético.

AP\texttt{\char`\_}Peripheral\texttt{\char`\_}RxTx é o projeto que implementa o funcionamento dos pontos de acesso. Ele atua concomitantemente como receptor e transmissor: recebe o ID do macaco monitorado em diversos horários até que o \emph{buffer} de mensagem a ser enviada ao ponto central esteja cheio, então envia as informações que recebeu conjuntamente com o horário calculado e o parâmetro de RSSI (usado no \emph{Raspberry Pi} para fazer o cálculo da distância).

Já o projeto AP\texttt{\char`\_}central\texttt{\char`\_}RxUART recebe as informações de monitoramento dos três pontos de acesso, e assim as envia por UART para o \emph{Raspberry Pi}.

\begin{figure}[ht]
  \centering
    \caption{ Funcionamento conjunto dos nós do sistema.}
    \includegraphics[scale=0.65]{sensortag-arquitetura}
  \centerline{\small{Fonte: autores}}
\end{figure}
\FloatBarrier

\section{Módulo Do Controlador}

Como módulo controlador escolheu-se uma placa \emph{Raspberry Pi} que deve estar conectada ao módulo central. Para evitar demais interferências do ambiente e com isso demais gargalos na comunicação, os dois módulos se comunicam de forma serial através de UART. No módulo controlador isso se dá pela biblioteca "serial" em python.

Ao receber as informações, o controlador deve formatar os dados das distâncias e \emph{timestamps} de cada medida tomada. Neste momento os valores de RSSI recebidos devem ser convertidos em distâncias em metros através do equacionamento obtido no item 2.4 deste documento e os \emph{timestamps} são ajustados para o tempo ao qual o controlador está conectado. Além disso, os valores da mensagem são checados para determinar se a mensagem está completa e coerente, ou seja, se não houve perda de dados durante os envios anteriores e é descartada em caso negativo.

Após receber e processar os dados o controlador deve enviá-los para o banco de dados no servidor. Já que não há problemas de armazenamento no controlador ele deve aguardar para enviar as informações para o banco assim que não houver mensagens sendo recebidas na conexão serial, já que seria demasiadamente custoso para o \emph{SensorTag} guardar e transmitir suas leituras.

A solução escolhida foi o uso da biblioteca \emph{MySQLdb} em \emph{Python}, com a qual os dados são inseridos diretamente no banco de dados utilizando \emph{SQL queries}. Como é boa prática, as mensagens são analisadas antes que de serem enviada para garantir que os dados se encaixam no que foi definido no banco. Caso haja problemas na conexão com o servidor elas são armazenadas no controlador até que ele consiga se reconectar.

O comportamento previamente descrito é esquematizado pela máquina de estados a seguir:

\begin{figure}[ht]
  \centering
    \caption{Máquina de estados representando o funcionamento do módulo controlador}
    \includegraphics[scale=0.65]{estados-raspberry}
  \centerline{\small{Fonte: autores}}
\end{figure}
\FloatBarrier

\section{Banco De Dados}

O modelo desenvolvido é relativamente simples e envolve somente os objetos físicos do contexto, de forma que fique claro que são mapeados: o \emph{target}, o ponto de acesso e a distância entre eles.

Pelo lado da segurança do sistema, foi utilizado um modelo padrão do \emph{Spring Security} que mapeia o usuário e suas permissões como pode ser visto no diagrama de classes a seguir.

\begin{figure}[ht]
  \centering
    \caption{Diagrama de classes gerado pela ferramenta de engenharia reversa do MySQL}
    \includegraphics[scale=0.8]{simios_db}
  \centerline{\small{Fonte: autores}}
\end{figure}
\FloatBarrier

A implementação do modelo foi feita em código com a declaração das entidades do sistema, que são interpretadas pela JPA. A configuração do banco de dados utilizado (MySQL) ao código permite que o JPA já construa e atualize as tabelas do banco quando necessário.

\section{Aplicação \emph{Web}}

A interface com o usuário prevê uma aplicação \emph{web} que permita a visualização dos gráficos e tabelas com dados dos animais.

Para isso, foi utilizado o \emph{framework Spring MVC} cuja estrutura consistia em considerar a necessidade de implementação de CRUDs para cada um dos objetos principais do sistema que o usuário deve ser capaz de gerenciar: o \emph{target}, o ponto de acesso e o usuário. Dessa forma, cada um destes objetos deve possuir:

\begin{itemize}
	\item views de cadastro, edição e listagem;
	\item um controlador, para mapear as requisições de Transferência de Estado Representacional (REST);
	\item um repositório;
	\item um serviço, para atuar no repositório.
\end{itemize}

A figura a seguir identifica o fluxo de cada uma das telas existentes no sistema. O protótipo das telas é exibido no Apêndice C.

\begin{figure}[ht]
  \centering
    \caption{Mapa do site}
    \includegraphics[scale=0.45]{mapa-do-site}
  \centerline{\small{Fonte: autores}}
\end{figure}
\FloatBarrier


\chapter{Testes e avaliação}

Foi determinada fundamentalmente a necessidade de avaliação do sistema sob duas frentes principais: consumo energético e precisão da posição obtida.

Para tanto, foi elaborado teste para a determinação da precisão na medida da distância entre o ponto de acesso e o \emph{target}. Dada a constante de propagação determinada como 2,95 (ver apêndice 2), foram comparados os valores exibidos pelo ponto de acesso e os medidos por uma trena, os quais estão disponíveis na tabela a seguir.

\begin{table}[ht]
\centering
\caption{Distâncias medidas pelo ponto de acesso e pela trena}
\vspace{0.5cm}
\begin{tabular}{l|cccccccc}
\hline
\makecell{Distâncias pela \\ trena (m)} & 0,5 & 1,0 & 2,0 & 3,0 & 4,0 & 5,0 & 6,0 & 7,0 \vspace{0.4cm}\\

\makecell{Distâncias \\ pelo AP (m)} &
\makecell{0,79 \\ 0,86 \\ 0,73 \\ 0,86 \\ 0,63 \\ 0,63 \\ 0,63 \\ 0,63 \\ 0,63 \\ 0,63 \\ 0,63} &
\makecell{1,00 \\ 1,08 \\ 1,08 \\ 1,00 \\ 1,00 \\ 1,00 \\ 1,08 \\ 0,68 \\ 1,26 \\ 1,00 \\ 1,00} &
\makecell{0,73 \\ 0,73 \\ 0,73 \\ 0,86 \\ 1,37 \\ 1,87 \\ 1,60 \\ 1,50 \\ 1,48 \\ 1,37 \\ 1,48} &
\makecell{8,22 \\ 2,76 \\ 3,22 \\ 2,55 \\ 2,18 \\ 2,18 \\ 2,18 \\ 2,02 \\ 2,18 \\ 2,55 \\ 1,60} &
\makecell{2,15 \\ 1,82 \\ 2,78 \\ 2,56 \\ 2,78 \\ 2,35 \\ 2,56 \\ 3,30 \\ 3,03 \\ 2,78 \\ 2,56} &
\makecell{3,59 \\ 2,56 \\ 4,26 \\ 3,03 \\ 3,91 \\ 8,43 \\ 7,74 \\ 5,05 \\ 6,53 \\ 5,05 \\ 6,53} &
\makecell{4,64 \\ 4,64 \\ 4,64 \\ 3,91 \\ 3,03 \\ 2,56 \\ 2,56 \\ 5,99 \\ 4,26 \\ 14,07 \\ 7,11} &
\makecell{7,04 \\ 5,15 \\ 6,51 \\ 10,40 \\ 7,61 \\ 13,14 \\ 8,90 \\ 3,77 \\ 4,08 \\ 2,98 \\ 4,76}
\vspace{0.4cm}\\

\makecell{Distância média \\ pelo AP (m)} & 0,69 & 1,02 & 1,24 & 2,88 & 2,61 & 5,15 & 5,22 & 6,76
\end{tabular}
\vspace{0.4cm}\\
\centerline{\small{Fonte: autores}}
\end{table}

\begin{table}[ht]
\centering
\caption{Erro quadrático médio e máximo para cada n}
\vspace{0.5cm}
\begin{tabular}{l|c}
\hline
Erro quadrático médio (m) & 1,52 \vspace{0.4cm}\\
Erro quadrático máximo (m) & 3,04
\end{tabular}
\vspace{0.4cm}\\
\centerline{\small{Fonte: autores}}
\end{table}

Dessa forma, foi possível determinar o erro quadrático obtido, que é aproximadamente 1,5m, como pode ser visto na tabela 3, que é um valor aceitável considerando a escala da área local de implementação. Entretanto, ainda é um erro alto para ambientes de cativeiro, cuja área é menor.

\begin{figure}[ht]
  \centering
    \caption{Gráfico das distâncias medidas}
    \includegraphics[scale=0.79]{graph-test-dist}
  \centerline{\small{Fonte: autores}}
\end{figure}

\begin{figure}[ht]
  \centering
    \caption{Gráfico do erro quadrático para as distâncias medidas em n=2.95}
    \includegraphics[scale=0.79]{graph-test-erro}
   \centerline{\small{Fonte: autores}}
\end{figure}

No que se refere ao consumo energético, o único teste prático que foi pensado exigiria expor uma bateria nova à exaustão e calcular o tempo tomado. No entanto, como dimensionamos o sistema para que possa durar semanas, este teste é pouco praticável e fica dentro do escopo secundário do projeto.


\chapter{Considerações Finais}

\section{Conclusões do Projeto de Formatura}
O presente projeto é composto por módulos bastante distintos e elaborados em diversos níveis de abstração: apresentando considerações acerca do hardware e firmware de placas a serem utilizadas até a elaboração do software de uma aplicação web de alto nível. Portanto ostenta intrincada configuração e integração de seus componentes resultando em uma gama de partes móveis que não são imunes a erros, apesar da cautela dos projetistas.

Contudo, todos os objetivos principais propostos que formam o cerne da funcionalidade: de obter informações acerca do animais de forma constante, de armazenar essas informações para partes interessadas e de apresentá-las de forma clara; foram atendidos pela presente implementação. O primeiro se correspondeu pela interação entre o módulo aplicado aos macacos e aquele aplicado às árvores que mantém contínua vigília sobre os animais enviando informações sobre sua localização. O segundo se alcançou na construção de um banco de dados contendo todos os dados observados de onde o sistema for instalado, cujo conteúdo disponibilizar-se-ia a quaisquer pesquisadores interessados para análise dos dados. O último se concretiza na disponibilidade dos dados em forma de mapas, gráficos ou tabelas, como for mais proveitoso, através de uma aplicação web de fácil utilização e que segue parâmetros de Experiência de Usuário como descrito por Preece, Sharp e Rogers em [12].

Em relação aos requisitos não funcionais levantados, destaca-se o cumprimento da exigência de generalização em relação ao objeto de observação do sistema; de fato, apesar de destinado à aplicação em símios de pequeno porte, com pouca ou nenhuma alteração o sistema pode ser facilmente adaptável para qualquer outro animal de comportamento similar ou a objetos inanimados que se queira investigar. Isto se deve à confirmação de que toda a base do sistema até o nível do banco de dados não exige configuração de qualquer parâmetro relacionado aos animais ou ao seu ambiente. Conclui-se então que o sistema é genérico o bastante para diversas aplicações, mas ressalva-se a necessidade de adaptar as informações coletadas para o novo domínio do problema.

Por outro lado, alguns aspectos do trabalho ficaram aquém dos resultados esperados. Como exemplo pode-se expor o fato dos módulos a serem carregados pelos macacos não atenderem à especificação de peso determinada, constituindo mais do que 10g, impedindo a observação de símios de menor porte. Isso não inviabiliza a monitoração pretendida, visto que a posição dos macacos ainda é assimilada e a monitoração da maioria dos animais não se torna irrealizável.

Outro exemplo pode ser feito da escolha de utilizar um nó central para condensar as informações do módulos dos pontos de acesso no coletor que, em contrapartida, prejudica a escalabilidade do sistema já que demanda que todos os pontos de acesso estejam a distância de transmissão do nó central. É possível resolver este ponto adaptando o sistema para que seja usado um protocolo de Redes Mesh, viabilizando a expansão da rede.

Além disso, em sua presente iteração o sistema não usufrui da acoplação de outros sensores embutidos no SensorTag como foi planejado, já que o foco residia na aquisição de informações sobre o posicionamento. Adaptações em todos os âmbitos do projeto devem ser feitas para possibilitar a captação e exibição de dados de outros sensores presentes na placa.

Dois pontos finais devem ser notados neste balanço. Primeiramente a dificuldade de instalação do sistema se mostrou uma questão complexa e imprevista em sua concepção. Tanto pela variação ambiental da constante necessária para calcular as distâncias a partir do RSSI de forma confiável, como pelas interferências no sinal e limites de distância aos quais a comunicação entre as placas estão sujeitas mostra-se que os requerimentos para o bom funcionamento do SIMIOS não são triviais. Isso apresenta dificuldade notável na implementação do sistema em reservas mais diversas ou próximas a ambientes urbanos, como é o caso do Instituto Butantan. Para tanto um estudo mais aprofundado da variação das constantes ao longo do espaço disponível e uma redundância maior na transmissão dos sinais seriam benéficos para o projeto.

Finalmente, não se deve ignorar os custos financeiros relacionados à possível implantação deste sistema em grande escala. É necessário considerar que para cada objeto de estudo é necessário adquirir uma placa SensorTag, assim como para cada incremento na área de observação, além disso há custo marginais na ampliação do banco de dados e servidores, o que resulta em um gasto nada trivial na expansão do sistema. Não obstante, dadas as alternativas de tecnologias atualmente disponíveis, apresentadas no segundo capítulo deste documento, a implementação proposta ainda se revela a opção de menor custo orçamentário.

\section{Contribuições}

Com o cumprimeto das funcionalidades necessárias neste sistema, acredita-se que o mesmo há de impactar de forma positiva os trabalhos de pesquisa em reservas nas quais se venha a instalar. Foi com o benefício de profissionais acadêmicos e de saúde animal que se baseou a elaboração das linhas guia deste projeto.

Em virtude dos estudos sobre as tecnologias presentes e das informações obtidas diretamente de profissionais da área, é possível dizer que, em comparação com as opções disponíveis de monitoramento de animais no mercado, as funcionalidades do SIMIOS atendem mais veementemente as necessidades correntes de primatologistas. Portanto o sistema contribui com uma solução mais bem adaptada aos profissionais de saúde e de comportamentalismo animal dadas as necessidades apresentadas destes grupos.

Atualmente o potencial de pesquisa nos campos de estudo da monitoração de animais, tanto no que diz respeito a pesquisas biológicas, como veterinárias quanto na esfera da Saúde Única é pouco explorado. Logo também é de interesse dos autores, não só satisfazer os requisitos do presente projeto, mas também dar destaque para tal nicho. Espera-se então que este trabalho sirva como inspiração e/ou ponto de partida para outros que almejam objetivos próximos através de metodologias similares. Adiciona-se a isso a expectativa de contribuições futuras como alternativas ao próprio SIMIOS, seguindo ou não os mesmos paradigmas ou as mesmas metodologias.

\section{Perspectivas De Continuidade}

Dadas as expectativas não cumpridas do trabalho, as primeiras propostas de continuidade pretendem atendê-las. Primeiramente a escalabilidade do sistema pode ser melhorada com o uso de outras técnicas de comunicação, como redes Mesh, na transmissão de dados entre módulos periféricos e pontos de acesso, permitindo uma configuração que consiga cobrir uma área maior. No mesmo âmbito, os sensores da placa podem ser configurados para enviar seus dados coletados juntamente com as mensagens sobre distância, com suporte nos módulos de envio e armazenamento é possível coletar uma gama maior de informações, limitada apenas à variedade de sensores na placa e ao tamanho máximo das mensagens transmitidas. Isso permitiria, por exemplo, a obtenção de informações sobre a temperatura de animais na região, uma das principais métricas no mapeamento da disseminação da febre amarela.

Para outros projetos que pretendem aproveitar os resultados deste, facilmente é possível adaptar o sistema para demais ambientes de uso, onde, além de problemas peculiares de cada ambiente, deve-se apenas atentar à determinação da constante eletromagnética do meio em questão e a eventuais desafios na transmissão de dados, como interferências comuns em comunicações sem fio. Além disso, a adaptação é igualmente fácil para outros objetos de estudo a serem observados, notando-se apenas que a mudança do domínio do problema pode acarretar mudanças em seus requisitos, necessitando, por exemplo, de continentes mais apropriados para as placas SensorTags considerando outras espécies de animais.

Dentre outros fatores de melhoria viáveis é possível destacar: o aumento do paralelismo e uso de threads tanto nos módulos de ponto de acesso, para garantir que mais leituras possam ser feitas em um mesmo local, quanto no módulo coletor, para evitar que este se torne um gargalo da aplicação; a realização de mais transmissões nas comunicações ou a exploração de mais componentes no protocolo para garantir a integridade das informações transmitidas; e a realização de mais medições de um mesmo módulo periférico em um curto espaço de tempo, cuja combinação dos valores permitiria obter resultados mais precisos sem a grandes perdas em termos de eficiência energética.
Por fim, se o compromisso entre fatores for reavaliado é possível dar sequência ao projeto utilizando outra tecnologia na obtenção da distância, como o uso de placas mais baratas, mas com menor variedade de sensores ou protocolos de comunicação. Dessa forma seria possível trocar a variedade de informações coletadas pela redução no custo. Ou contrariamente, pode-se adquirir placas mais caras, porém com alcance superior ou faixa de transmissão menos suscetível a falhas. Trocando neste caso um aumento no custo financeiro do projeto por uma precisão e confiabilidade maior dos dados. De qualquer forma que se pense refatorar o sistema, partes dele, como o banco de dados ou a aplicação web nos casos exemplificados, ainda podem ser, e recomenda-se que sejam, reaproveitadas.


% ========== Referências ==========
% --- IEEE ---
%	http://www.ctan.org/tex-archive/macros/latex/contrib/IEEEtran
%\bibliographystyle{IEEEbib}

% --- ABNT (requer ABNTeX 2) ---
%	http://www.ctan.org/tex-archive/macros/latex/contrib/abntex2
\bibliographystyle{abntex2-num}

\bibliography{referencias-bibliograficas}{}


% ========== Apêndices (opcional) ==========
\apendice
\chapter{Entrevista com a profª Drª Cristiane Schilbach Pizzutto}

Para agregar ao conhecimento interdisciplinar necessário para a projeção deste trabalho, foi realizada entrevista com a professora Cristiane Pizzutto, da Faculdade de Medicina Veterinária da USP, cuja redação foi documentada a seguir com o consentimento da doutora.

\textbf{09 maio 2018 - São Paulo, SP}

P: Quantos anos você tem?

R: Quarenta e cinco.

P: Qual a sua área de atuação?

R: Eu trabalho com a  parte de animais silvestres em cativeiro. Mais especificamente, eu trabalho com enriquecimento orientado: avalio o estresse desses animais, mas também todo um monitoramento comportamental e endócrino deles em função dos ambientes que eles vivem e também o recurso de animais silvestres, muito voltado para a conservação. Hoje o cativeiro tem um papel importante na conservação desses animais, por isso muitas espécies a gente tem que estar preocupado em reproduzir para depois tentar reintroduzir no meio ambiente.

P: Quanto aos animais, essa monitoração é feita em laboratórios?

R: No laboratório a gente só faz a parte de dosagens hormonais dos animais.


P: Como se chama o lugar dos animais?

R: Cativeiro, zoológicos, aquários. Nesses espaços, a gente fala que é o recinto desses animais. O cativeiro nada mais é do que a privação da liberdade desses animais, não podem fugir, o que não está em vida livre está cativo. Eu trabalho com estes animais.


P: Quantos animais você acha que acaba monitorando? Tem o controle de animais que controla ao longo do tempo ou você vai ao local e tem alguns animais e você conhece eles na hora?

R: Então, eu não trabalho com animais de vida livre, é diferente de uma pesquisa de quem trabalha em campo e quem trabalha com animais de cativeiro. No meu caso, eu conheço os animais e todos estão envolvidos no projeto que eu trabalho. Meu acompanhamento é quase individual. Eu tenho como saber o que acontece com ele do ponto de vista comportamental e hormonal.


P: Quantos animais são monitorados em um projeto?

R: Depende. Quando se trabalha com animal silvestre, a gente trabalha com um N reduzido, mas geralmente é aquela quantidade daquele cativeiro. Por exemplo, com aves geralmente a gente trabalha com mais animais: num projeto com araras temos mais do que 25 araras, mas quando trabalhamos com espécies mais raras, como a onça pintada, temos só uma ou duas em cativeiro. Quase nunca passa de trinta, nosso número mais comum é 1, o que dificulta fazer estatísticas e com uns 7 já estamos dando pulos de alegria.


[Estamos pensando em 10 saguis.
Maravilhoso.]


P: Quanto tempo você dedica às tarefas de ir até os animais? Quanto tempo demora isso?

R: Depende. No meu trabalho, quando a gente mexe com a parte comportamental. Em qualquer trabalho com comportamento, exige ao menos vinte a trinta horas de observação, que eu tenho que dividir ao longo de um período para obter uma estatística sensata. Além disso, tenho que determinar qual período do dia eu posso observar esses animais, por que têm momentos do dia que eles estão mais ativos e momentos que eles estão menos. Porque eu quero a maior quantidade de informações comportamentais pra mim dos períodos mais ativos. Isso também é diluído ao longo da metodologia dos trabalhos. Como eu trabalho com mudanças ambientais, eu faço sempre o antes e o depois: Eu faço de trinta a quarenta horas antes e de trinta a quarenta horas depois. Por isso eu tenho que dar uma boa lapidada neles que eles vão ter que ter paciência e tempo, porque demanda concentração nesses trabalhos. Quando eu faço coleta de material para análise hormonal é coisa rápida, às vezes trabalho com fezes que nem coloco a mão no animal; é um método novo que é pouco invasivo. Mas se for trabalhar com sangue tem que dar anestesia.


P: Essas trinta a quarenta horas são de observação direta?

R: Direta, tem que estar na frente do animal e fazer registros. A gente faz o que se chama de histogramas, que são formas de análise do comportamento. Tem vários métodos: posso observar o animals de forma instantânea, de tempos em tempos eu faço o registro; ou posso fazer um registro contínuo onde eu observo o dia inteiro sem pausa. Tem várias formas. Eu acho que o sistema de vocês o mais interessante é o envio de informações instantâneas. 


P: Essa é minha outra pergunta, você acha que as informações que você acaba indo a campo monitorar dos animais, se você acha que essas informações podem ser captadas e enviadas para laboratório? Se isso ajuda?

R: Não sei se vocês conseguem esse tipo de informações, porque não sei se sua placa consegue mandar a informação de que o animal tá comendo, o que eu posso ver quando tô observando o animal. Isso me interessa porque cativeiros geralmente tem o histórico de ser muito ruins, eles não atendem às necessidades dos animais. Por exemplos alguns primatas comem no chão, outros vivem em extratos de 10 metros, outros de 15 metros. Com a observações em cativeiro, eu consigo fazer essa avaliação, olha eu vejo um animal que se alimenta na faixa de 2-3 metros, vocês podem não conseguir fazer essas observações e essas são informações valiosas pra mim. Não sei se vocês vão conseguir ter acesso a essas informações e enviá-las pro laboratório.


P: Para esse tipo de coisa, você acha válido ter câmeras?

R: Sim.


P: Vocês têm câmeras em alguns cativeiros já?

R: Instaladas, sim, mas elas são fixas e eles filmam os animais. Mas vocês com esse sistema, o ideal seria você ter uma câmera para enxergar o que o animal tá fazendo.


P: Quais são procedimentos padrões, tem alguma rotina fixa para esse tipo de pesquisa ou depende do animal ou do cativeiro?

R: Depende de várias coisas: da instituição, que tem que liberar o acesso pra gente poder observar os animais; e como o projeto começa, a gente tem que seguir a metodologia. Vamos supor que eu tenha 6 meses pra fazer a linha de base que é fazer as observações primárias para descobrir qual momento do dia o animal tá mais ativo e a gente divide o número de horas durante 6 meses, cada dia que você for lá você vai observar 1 hora nesse período de maior atividade porque nesses 6 meses tem que fazer as quarenta horas de registro.


P: A avaliação inicial do período de atividade é feita por observação direta também?

R: Sim, mas é diferente de como a gente observa. Por exemplo, se eu tenho 5 animais num ambiente, a cada 5 minutos durante um dia inteiro, por 3 dias, ela vai observar e falar que eu tenho 2 animais ativos às 7:00 e 3 inativos, ai ela vai fazendo os registro de 15 em 15 min para fazer um gráfico de atividade e inatividade, e a gente consegue ver qual momento do dia a gente tem mais e menos animais ativos. Ativo ele ta fazendo qualquer coisa, inativo é que ele ta deitado sem fazer absolutamente nada, porque, primatas em especial, são animais que têm variabilidade comportamental, eles fazem absolutamente tudo ao mesmo tempo, quando a gente vai fazer registro a gente fica louca de tanta informações que eles passam pra nós.


P: Esse dados... tem uma formatação dos dados, vocês usam algum tipo de programa?

R: A gente já tentou usar um que o nome eu não lembro… mas parece um aparelho que registra água e luz. Mas eu não consegui me adaptar com ele, porque ele me fala assim: agora tá na hora de registrar, mas às vezes o animal tá representando um comportamento que não tá cadastrado no sistema e o sistema não aceita. Prefiro fazer tudo no papel e os alunos no final do dia contabilizam os comportamentos e jogam em uma planilha do excel, porque esse aparelho tem que ser pré programado para todos os comportamentos que o animal pode ter e nem sempre o animal executa o comportamento.

Seria melhor alguma coisa mais aberta ou personalizada. Seria fantástico.
A gente trabalha com sigla, porque tem que ser muito rápido, tipo, comendo depois a gente sabe que é CO.


P: Como é feito para localizar o animal dentro de cativeiro? Como é feito para identificar o animal?

R: Muitos a gente consegue saber quem é quem, principalmente por características externas, às vezes tem um defeito na orelha ou uma pelagem diferente e primata tem uma feição muito diferente entre eles, é fácil.

Mas por exemplo, em um trabalho que estou orientando no aquário de São Paulo com cangurus, e aí é uma coisa difícil de identificar e a gente vai pro coletivo e não pro indivíduo. A gente registra pelo grupo e fazemos um método chamado de scan. Os registros podem ser contínuos ou instantâneos, por intervalo de tempo, mas nesses caso eu preciso fazer um scan do grupo e instantâneo, já que contínuo é impossível, fazendo um registro do que cada animal está fazendo sem identificar quem é quem.


P: No contínuo você faz um log a todo momento de todas as atividades independente do tempo?

R: Eu uso o contínuo quando eu quero fazer um registro detalhado de como o animal come, ai eu olho o animal: pegou a comida, levou até a boca, devolveu, engoliu, o registro de como ele come. No meu caso é mais interessante fazer o instantâneo, porque eu foco no tipo de comportamento e acabo tendo milhares de registros no final de um dia, imagina no final de um projeto. Eu cheguei a ter 170.000 em uma planilha de excel com um gorila que eu trabalhei durante 8 anos e aí eu consigo ter frequência de ocorrências. Eu trabalho com quantidade de comportamento, eu transformo um em milhares de dados (informações).

Fica coisa pra caramba, aí eu dou uma driblada na estatística, por que 1 não é estatística, mas com 170.000 já dá pra fazer estatística. A gente faz o que a gente pode com animais tão ameaçados. Não tem, não adianta querer ter. Isso é difícil de convencer a comunidade científica, nem todo mundo é da área e as pessoas não entendem a importância de trabalhar com um indivíduo. Não adianta querer ter pelo menos 2, não tem.


P: Voltando, sobre o programa que você falou que trabalhava com anotação no papel, tem algum outro instrumento que você usa?

R: Uso muito uma câmera fotográfica para filmar porque o comportamento é dinâmico. Nem sempre as pessoas entendem quando a gente descreve e fazer registro é muito importante. E às vezes binóculo, só para enxergar melhor os comportamentos. Para registro: foto filmagem e papel. Basicamente isso.


P: Os animais se comportam muito diferente quando você observa eles?

R: Sim, muitos se estressam com a presença de humanos, por isso a gente faz uma etapa de socialização, então a gente fica um dia por perto pra ele ver a gente e se acostumar. Quando ele se acostuma com a gente lá e faz as observações. O que eles têm é muito problema de comportamento pro cativeiro que eles tão, aí em cativeiro é muito ruim e não atende as atividades deles do dia a dia e eles têm comportamentos anormais, estereotipados. Aí tem um monte de coisa, é com isso que eu trabalho. Aí eu entro com a técnica de enriquecimento ambiental que é pra restaurar os comportamentos normais que eles tem e melhorar a qualidade de vida deles e quando isso acontece, a gente tem sucesso na reprodução. Esse é o segredo de manter zoológico hoje, não é pra manter só por manter, a gente tem que dar qualidade de vida pra eles pra que eles justifiquem o animal estar preso.


P: Seu maior interesse é aproximar o comportamento que eles têm em cativeiro pro que eles têm fora?

R: Isso, quero resgatar comportamentos que eles perdem. Eles perdem certos comportamentos em cativeiro: eles tem tanto estresse que acabam tendo mais comportamento anormais do que naturais e típicos da espécie e o objetivo é resgatar esses comportamentos naturais.


P: Eles não ficam todos em reservas?

R: Não temos reservas para todos. Eu vou falar pra vocês, é triste isso, mas temos ativistas que falam que temos que abrir portas do zoológico e soltar. Eles vão morrer se a gente soltar. Não tem cadeia alimentar que suporte esses animais: os ambientes tão destruídos. Se a gente fizer uma soltura em massa, a gente não vai ter suporte alimentar para todos. A gente vai matá-los com certeza. Não tem o que fazer.  

A questão é essa: tem que considerar a situação que o animal tá vivendo e que o ambiente natural tá vivendo. Há 10 anos, eu não defendia tanto cativeiro quanto eu defendo hoje. Agora eu preciso defender a existência de cativeiro, porque nós vamos perder os últimos membros das espécies. Trabalhar pela preservação do que nós temos em prol da conservação da natureza.


P: Esse processo de socialização com os animais em cativeiro, como ele é feito? Leva muito tempo para eles se acostumarem? 

R: Eu comecei com um trabalho com um gorila que tinha no zoológico de São Paulo, que era solitário por muitos anos. Quando eu voltei desse estágio nos EUA, que eu vi que eles faziam essa questão de enriquecimento ambiental e tudo, que aqui no Brasil não tinha, e foi inédito. Eu comecei observando ele a distância. Só observar, até uma hora que eu percebi que o animal me observava, não mais eu observava ele, ai eu peguei e sentei na frente da grade dele e fique fazendo uma aproximação porque eu queria condicioná-lo. Esse processo demorou 6 meses. Porque era um gorila, um exemplar raro; que você não pode olhar no olhos se não chama ele para um desafio, então toda vez que eu entrava na frente dele eu tinha que ficar submissa até que eu treinei ele para que eu pudesse fazer procedimentos veterinários. Aí eu fazia supressão e eu colocava o estetoscópio nele, ele abria a boca pra mim e eu inspecionava os dentes dele. Tudo isso ele faz de forma voluntária. Aí você começa a fazer um trabalho que ele quer fazer com você porque o gorila é um animal social e, como ele tá sozinho, interagir com você pra ele é muito bom. Então começa a interagir com animal e começo a tirar proveito dessa situação: não preciso anestesiar ele pra fazer um procedimento rápido, eu peço para ele abrir a boca e ele abre, sabe. Acaba sendo uma facilidade de manejo. 

Para observação não. Você começa a observar o animal, fica ali algum tempo, habituando ele com a sua presença. Em uns 20 dias você faz a habituação.


P: Você vira um hábito dele então?

R: Sim, passa a ser normal pra ele.


P: Você já teve experiência com tratamento de reserva?

R: Não, reserva não.


P: Por que?

R: Sempre trabalhei com cativeiro. Sempre me preocupei com melhorar a condição dos animais em cativeiro, mudar demais a situação deles. Minha pesquisa sempre foi pra isso.

Porque acho importante. Eu trouxe isso pra veterinária, não temos o hábito de trabalhar com isso, tem biólogos que trabalham com isso, mas não veterinários. Essa foi outra barreira que eu encontrei porque veterinário não trabalha com acompanhamento de comportamento, que é mais coisa de biólogo do que de veterinário, ou seja nem posso mais falar isso, porque eu to brigando muito para que seja uma questão da veterinária. Porque a gente tem que estudar o comportamento do animal pra gente poder tratar, então foi uma coisa nova na veterinária. Meu desejo sempre foi cativeiro pra ajudar. 

Tudo começou com esse gorila porque me incomodava a situação dele desde pequena, eu ia ao zoológico e vendo a situação dele eu decidi ser veterinária. Com 7 ou 8 anos: “Pai, quero cuidar desse gorila. Não gosto da situação que ele está”. Quando acabei a minha graduação, montei um projeto pra ele e fiquei com ele por 11 anos. Ele me botou a ideia de trabalhar com bem estar em cativeiro, então, não monitoro animais em vida livre.


P: Tem alguma coisa no seu trabalho que você tenha alguma dificuldade, ou seja, que a gente poderia fazer pra te ajudar com a nossa área? Por exemplo, como você usava o excel de ferramenta, tem alguma outra ferramenta que te ajude/ajudaria?

R: Se houvesse um sistema aberto pra mim, pra fazer registros, isso seria fantástico. O que seria maravilhoso, se tivesse uma placa que coletasse amostras pra mim de tempos em tempos, tipo uma agulha de insulina que coletasse amostras de sangue de animais em vida livre, ou que coletasse tecidos subcutâneos. Seria muito bom. Mas se a placa de você já fosse capaz de medir frequência cardíaca e temperatura corporal isso já seria muito bom, ou até pressão do animal. E para mim me interessa saber a proximidade também.

Vou amarrar isso tudo com a importância da Saúde Única: hoje, a gente trabalha com a saúde do homem, do animal e do ambiente. Então quando vocês colocam um dispositivo para monitorar a febre amarela, vocês tão monitorando a saúde do ambiente e do homem. Aí vocês podem trabalhar no TCC de vocês a questão de Saúde Única e a importância do projeto de vocês com esse conceito. Porque hoje em dia, a gente tá preocupado com a intersecção dessas saúdes. Não dá mais pra desconectar. O que a gente tem com a febre amarela: Desastre de Mariana que acabou com a população de anfíbios, que fez uma proliferação de mosquitos e houve um desequilíbrio total com a doença, então um aspecto ambiental afetou a saúde do homem e do animal, o homem acaba sendo vítima.

Faria uso de uma ferramenta dessas em benefício dos animais e da profissão. O que acontece hoje é que a gente não consegue salvar os animais a tempo. Hoje em dia, a gente não consegue ter condições de ter tantas informações pra conseguir salvar os animais. A situação é difícil e quem trabalha chega a pôr do bolso os recursos.


P: Sobre a exibição das informações, como você faz pra fazer gráficos sobre as informações coletadas? Como se apresenta os dados?

R: Bom, fazendo uma análise estatística, mas dependendo do trabalho que se está fazendo, precisa de um teste específico. Por exemplo, a onça pintada, que está no gargalo da ameaça, e tem poucos animais em vida livre, ela vive sozinha: só encontra o parceiro no momento da reprodução. Em cativeiro a gente tem um monte, mas todos são velhos, castrados, que não se reproduzem, então meus alunos estão fazendo trabalho de microbiologia com a onça e fazer transferência de embrião, mas eu precisava entender comportamento de cópula para ter informação básica. E a gente achou um criadouro no pantanal que avaliou o comportamento desse casal fazendo 210 vídeos de registro de cópula. E como faz as estatísticas? Por relato, mas os trabalhos científicos querem estatística. Então a gente fez um Fisher Test e com ele deu certo. Mas entende? A gente não tem como fazer estatística, só o relato. Mas a comunidade científica exige e a gente deixa de publicar uma informação importante, que vai ajudar outros projetos, porque se tem uma régua muito alta de exigência e a gente não consegue publicar.

 
P: Tem algum jeito de ajudar vocês a obter as estatísticas, para facilitar isso?

R: Então, depende muito do seu projeto. Podem ter diversas esferas de análise.


P: Sobre o colar de rádio, ele pega os sinais vitais?

R: Pega: batimento cardíaco, por exemplo.


P: Mas ele já pega outras informações?

R: Os de felinos captam frequência cardíaca e temperatura, mas outros que são mais baratos são só pra passar informação de onde o animal está. Outros pegam temperatura também.
\chapter{Determinação da constante de propagação}

A constante de propagação do ambiente determinada pela Texas Instruments deve ser definida empiricamente. Portanto, foram tomados dez valores RSSI para distâncias conhecidas, medidas com uma trena, para que pudessem ser verificados os pontos críticos em que pudéssemos nos manter dentro do erro previsto. Além disso, foi medido RSSI consistente de -60 dB para 1 metro de distância.

\begin{table}[ht]
\centering
\caption{Intensidades observadas para cada distância medida com trena}
\vspace{0.5cm}
\begin{tabular}{l|ccccccc}
\hline
Distância (m) & 0,5 & 1,7 & 2,5 & 5 & 7,5 & 10 & 12,5 \vspace{0.4cm}\\

RSSI (dB) & 
\makecell{-56 \\ -57 \\ -62 \\ -56 \\ -55 \\ -58 \\ -61 \\ -62 \\ -63 \\ -63 \\ -63 \\ -63 \\ -63} &
\makecell{-69 \\ -63 \\ -63 \\ -64 \\ -64 \\ -65 \\ -69 \\ -69 \\ -69 \\ -69 \\ -69 \\ -69 \\ -65} &
\makecell{-72 \\ -69 \\ -62 \\ -62 \\ -71 \\ -70 \\ -64 \\ -63 \\ -64 \\ -63 \\ -64 \\ -63 \\ -62} &
\makecell{-86 \\ -79 \\ -77 \\ -79 \\ -81 \\ -82 \\ -82 \\ -81 \\ -81 \\ -80 \\ -80 \\ -81 \\ -81} &
\makecell{-85 \\ -90 \\ -86 \\ -85 \\ -84 \\ -84 \\ -85 \\ -90 \\ -91 \\ -90 \\ -87 \\ -87 \\ -86} &
\makecell{-81 \\ -81 \\ -82 \\ -82 \\ -81 \\ -82 \\ -84 \\ -84 \\ -85 \\ -84 \\ -85 \\ -87 \\ -86} &
\makecell{-84 \\ -89 \\ -96 \\ -96 \\ -90 \\ -93 \\ -90 \\ -95 \\ -95 \\ -94 \\ -98 \\ -96 \\ -97}
\vspace{0.4cm}\\

RSSI médio (dB) & -60,15 & -66,69 & -65,30 & -80,77 & -86,92 & -83,38 & -93,30
\vspace{0.4cm}\\

n & -0,05 & 2,90 & 1,33 & 2,97 & 3,08 & 2,34 & 3,04
\end{tabular}
\end{table}

Assim, tomando o RSSI médio para cada distância, são listados os coeficientes de propagação ideais de cada medida. Tomando a moda de aproximadamente 2,95 e a média de 2,6, foram calculadas as distâncias que seriam medidas para ambos os coeficientes e, então, calculado o erro quadrático em cada ponto.

\begin{figure}[ht]
  \centering
    \includegraphics[scale=0.916]{graph-dist}
  \caption{Gráfico das distâncias calculadas para cada n (Fonte: autores)}
\end{figure}

\begin{figure}[ht]
  \centering
    \includegraphics[scale=0.92]{graph-erro}
  \caption{Gráfico do erro quadrático para cada n (Fonte: autores)}
\end{figure}

\begin{table}[ht]
\centering
\caption{Erro quadrático médio e máximo para cada n}
\vspace{0.5cm}
\begin{tabular}{l|cc}
\hline
n & 2,95 & 2,6 \vspace{0.4cm}\\
Erro quadrático médio (m) & 2,82 & 4,06 \vspace{0.4cm}\\
Erro quadrático máximo (m) & 7,84 & 7,84
\end{tabular}
\end{table}

Por fim, concordou-se em utilizar o valor de n=2,95 por apresentar menor erro.

O processo realizado para a determinação do coeficiente de propagação poderia ser automatizado com algoritmos que inclusive considerassem mais valores caso fosse interessante expandir a aplicação dessa forma, por exemplo, anexando sonares aos pontos de acesso.
\chapter{Fluxo de telas da aplicação}

As figuras a seguir ilustram o protótipo de telas da aplicação \emph{web}.

\begin{figure}[ht]
  \centering
    \caption{Protótipo da \emph{homepage} não autenticada e tela de entrada}
    \includegraphics[scale=0.8]{fluxo-telas-1}
  \centerline{\small{Fonte: autores}}
\end{figure}

\begin{figure}[ht]
  \centering
    \caption{Protótipo da \emph{homepage} autenticada e da tela de listagem de usuários} 
    \includegraphics[scale=0.9]{fluxo-telas-2}
  \centerline{\small{Fonte: autores}}
\end{figure}

\begin{figure}[ht]
  \centering
    \caption{Protótipo das tela de cadastro de usuários e listagem de símios}
    \includegraphics[scale=0.9]{fluxo-telas-3}
  \centerline{\small{Fonte: autores}}
\end{figure}

\begin{figure}[ht]
  \centering
    \caption{Protótipo da tela do mapa}
    \includegraphics[scale=0.9]{fluxo-telas-4}
  \centerline{\small{Fonte: autores}}
\end{figure}

\chapter{Memorial de Cálculo do Período do Timer}

Uma das estratégias de temporização do módulo de \emph{target} mencionado no capítulo 6 é o uso de um periférico \emph{timer} de propósito geral, disponível pelo Microcontrolador presente nas placas do \emph{SensorTag} \cite{datasheet}.

De maneira geral, um \emph{timer} funciona como um contador, podendo ser de modo up, down ou uma combinação de ambos, periódico ou “one-shot”, Modulação por Largura de Pulso (PWM - \emph{Pulse Width Modulation}), entre outros. No contexto deste trabalho, o uso do \emph{timer} seria feito de modo periódico, e o MCU prevê o funcionamento de seus \emph{timers} apenas no modo up. Portanto, um timer no \emph{SensorTag}, essencialmente, funciona como um contador de 0 até um valor especificado por seu programador, podendo repetir isto de modo periódico ou não.

O periférico é programado por meio de um \emph{driver}, cuja API é fornecida pela própria TI \cite{shibata}. Um exemplo de codificação seria o seguinte:

\begin{lstlisting}
GPTimerCC26XX_Handle hTimer;
void timerCallback(GPTimerCC26XX_Handle handle, GPTimerCC26XX_IntMask interruptMask) {
       // interrupt callback code goes here.
       // Minimize  processing in interrupt.
}
void taskFxn(UArg a0, UArg a1) {
     GPTimerCC26XX_Params params;
     GPTimerCC26XX_Params_init(&params);
     params.width = GPT_CONFIG_16BIT;
     params.mode = GPT_MODE_PERIODIC_UP;
     params.debugStallMode = GPTimerCC26XX_DEBUG_STALL_OFF;
     hTimer = GPTimerCC26XX_open(CC2650_GPTIMER0A, &params);
     if(hTimer == NULL) {
     Log_error0("Failed to open GPTimer");
     Task_exit();
}
     Types_FreqHz freq;
     BIOS_getCpuFreq(&freq);
     GPTimerCC26XX_Value loadVal = freq.lo / 1000 - 1;
     GPTimerCC26XX_setLoadValue(hTimer, loadVal);
      GPTimerCC26XX_registerInterrupt(hTimer, timerCallback,
        GPT_INT_TIMEOUT);
      GPTimerCC26XX_start(hTimer);
      while(1) {
           Task_sleep(BIOS_WAIT_FOREVER);
      }
}
\end{lstlisting}

Um breve comentário acerca do código acima: há a declaração inicial de uma estrutura de referência a um \emph{timer}, bem como os parâmetros que definem seu funcionamento (16 bits, periódico, sua função de callback, dentre outros). O parâmetro loadVal é o que define o valor até o qual o \emph{timer} conta. Ele deve ser escolhido para controlar as chamadas à função de callback(), a qual ocorre sempre que o contador chega em seu valor máximo (definido em loadVal).

Considerando que o contador é atualizado sincronizadamente com o clock do sistema de 48MHz (ou seja, a cada 1/48M segundos), e que ele deve ser atualizado (loadVal + 1), então o tempo para cada período do clock pode ser calculado como:

\begin{equation}
PeriodoTimer = \dfrac{loadVal + 1}{48MHz}
\end{equation}

Fixando-se então o período desejado para o \emph{timer}, o valor de loadVal deve ser:

\begin{equation}
loadVal = PeriodoTimer \times 48MHz - 1
\end{equation}



% ========== Anexos (opcional) ==========
%\anexo
%\chapter{Alpha}
%\chapter{}



\end{document}
