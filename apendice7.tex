\chapter{Estudo para aplicação do projeto em Realidade Aumentada}

Este trabalho de projeto de formatura teve parte de seu desenvolvimento realizado na disciplina de Interface Humano Computador. Na ocasião, foi necessário realizar um projeto de estudo de usuário como parte da avaliação na disciplina e com isto surgiu-se a possibilidade de se implementá-lo utilizando-se Realidade Aumentada (RA).\\

Para este projeto de formatura, uma pesquisa de estudo de usuário se justifica por buscar a melhor experiência de usuário para os biólogos primatologistas, os quais são os usuários finais deste sistema.
No que se refere a uma implementação em RA, a ideia abrange a atuação do pesquisador em campo, sendo que RA o auxilia então a localizar os animais monitorados. O dispositivo utilizado deve ser capaz de mostrar uma projeção do ambiente com sinalização da localização dos animais e detalhes sobre.

\section{Stakeholders}

São usuários primários veterinários e pesquisadores que realizam trabalho de campo em reservas naturais, que usarão o sistema com frequência. Usuários secundários são estudantes de biologia ou veterinária que ocasionalmente realizem trabalho de campo acadêmico. Demais usuários poderiam ser visitantes de parques naturais interessados em ver animais silvestres.

\section{Papéis e Variáveis de Perfil}

Foram identificados os seguintes papéis:

\begin{enumerate}
\item Veterinário - trabalha diretamente com o animal e se preocupa essencialmente com o bem estar e saúde do mesmo (mais preocupado com informações que reflitam sua condição corporal);
\item Pesquisador (biólogo/psicólogo/estudante) - trabalha distanciado do animal, se preocupando em obter informações relacionadas ao comportamento do mesmo;
\item Visitantes (turistas) - interessados em localizar e interagir o máximo possível com os animais.
\end{enumerate}

O pesquisador é escolhido como usuário primário para ser alvo do estudo. Para este papel, as variáveis relevantes levantadas são:

\begin{itemize}
\item Formação acadêmica (curso e grau): identifica a profundidade da pesquisa que é realizada
\item Qual reserva o pesquisador trabalha: dimensiona o tamanho e a quantidade de animais que são monitorados;
\item Há quanto tempo a pessoa trabalha com esse tipo de pesquisa: identifica o grau de experiência do usuário;
\item Quanto tempo do seu ofício é dedicado ao monitoramento dos macacos: identifica a ênfase que é dada à tarefa.
\end{itemize}

\section{Necessidades}

As perguntas elaboradas que expressam a motivação deste trabalho são:

\begin{enumerate}
\item Quais são alguns dos procedimentos padrão do pesquisador/veterinário de campo? O que é feito e onde?
\item Como é feito para localizar um macaco específico?
\item Que tipos de equipamentos e ferramentas são manipulados?
\item Quais informações são necessárias?
\item Que tipo de planejamento prévio é realizado para executar bem essa atividade?
\item A forma como tudo isso é feito te agrada? Existem dificuldades?
\end{enumerate}