\chapter{Memorial de Cálculo do Período do Timer}

Uma das estratégias de temporização do módulo de target mencionado no capítulo 6 é o uso de um periférico timer de propósito geral, disponível pelo Microcontrolador presente nas placas do sensortag \cite{datasheet}.

De maneira geral, um timer funciona como um contador, podendo ser de modo up, down ou uma combinação de ambos, periódico ou “one-shot”, Pulse Width Modulation (PWM), entre outros. No contexto deste trabalho, o uso do timer seria feito de modo periódico, e o MCU prevê o funcionamento de seus timers apenas no modo up. Portanto, um timer no sensortag, essencialmente, funciona como um contador de 0 até um valor especificado por seu programador, podendo repetir isto de modo periódico ou não.

O periférico é programado por meio de um driver, cuja API é fornecida pela própria Texas Instruments \cite{shibata}. Um exemplo de codificação seria a seguinte:

\begin{lstlisting}
GPTimerCC26XX_Handle hTimer;
void timerCallback(GPTimerCC26XX_Handle handle, GPTimerCC26XX_IntMask interruptMask) {
       // interrupt callback code goes here.
       // Minimize  processing in interrupt.
}
void taskFxn(UArg a0, UArg a1) {
     GPTimerCC26XX_Params params;
     GPTimerCC26XX_Params_init(&params);
     params.width = GPT_CONFIG_16BIT;
     params.mode = GPT_MODE_PERIODIC_UP;
     params.debugStallMode = GPTimerCC26XX_DEBUG_STALL_OFF;
     hTimer = GPTimerCC26XX_open(CC2650_GPTIMER0A, &params);
     if(hTimer == NULL) {
     Log_error0("Failed to open GPTimer");
     Task_exit();
}
     Types_FreqHz freq;
     BIOS_getCpuFreq(&freq);
     GPTimerCC26XX_Value loadVal = freq.lo / 1000 - 1;
     GPTimerCC26XX_setLoadValue(hTimer, loadVal);
      GPTimerCC26XX_registerInterrupt(hTimer, timerCallback,
        GPT_INT_TIMEOUT);
      GPTimerCC26XX_start(hTimer);
      while(1) {
           Task_sleep(BIOS_WAIT_FOREVER);
      }
}
\end{lstlisting}

Um breve comentário acerca do código acima: há a declaração inicial de uma estrutura de referência a um timer, bem como os parâmetros que definem seu funcionamento (16 bits, periódico, sua função de callback, dentre outros). O parâmetro loadVal é o que define o valor até o qual o timer conta. Ele deve ser escolhido para controlar as chamadas à função de callback(), a qual ocorre sempre que o contador chega em seu valor máximo (definido em loadVal).

Considerando que o contador é atualizado sincronizadamente com o clock do sistema de 48MHz (ou seja, a cada 1/48M segundos), e que ele deve ser atualizado (loadVal + 1), então o tempo para cada período do clock pode ser calculado como:

\begin{equation}
PeriodoTimer = \dfrac{loadVal + 1}{48MHz}
\end{equation}

Fixando-se então o período desejado para o timer, o valor de loadVal deve ser:

\begin{equation}
loadVal = PeriodoTimer \times 48MHz - 1
\end{equation}
