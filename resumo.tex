\begin{resumo}
A execução deste trabalho visa a projeção e implementação de um sistema capaz de auxiliar pesquisadores de saúde e biologia animal no monitoramento comportamental de uma população em estudo. Para tal, foi elaborado um wearable capaz de obter a localização de animais em observação inseridos em ambiente controlado no qual uma rede de sensores é integrada. Tal informação é enviada para uma plataforma onde os dados poderão ser analisados pelo pesquisador remotamente. O desenvolvimento deste sistema tem como finalidade atender às necessidades de monitoramento de símios na reserva natural do Instituto Butantã da Universidade de São Paulo. Contudo, pretende-se que sua aplicação seja possível em distintas reservas, para a observação de diferentes animais e que possa eventualmente ser expandido para captar demais variáveis de controle relevantes para o biólogo ou veterinário. Finalmente, o sistema obtido cumpre os requisitos definidos e demonstra escalabilidade.

\textbf{Palavras-Chave} -- Monitoramento animal remoto, Redes de sensores sem fio, Embarcado.
\end{resumo}
