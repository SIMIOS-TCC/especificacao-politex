\begin{resumo}

Este trabalho envolveu a projeção e implementação de um sistema capaz de auxiliar pesquisadores de saúde e biologia animal no monitoramento comportamental de uma população de animais em estudo dentro de reservas. Para tal, foi elaborada uma arquitetura composta por uma rede de sensores, um banco de dados e uma aplicação \emph{web}. As informações sobre a localização dos animais foram obtidas pela rede de sensores e armazenadas no banco de dados, onde eram acessadas pela aplicação \emph{web} para serem analisadas por pesquisadores. O desenvolvimento deste sistema teve como finalidade atender às necessidades de monitoramento de símios na reserva natural do Instituto Butantan da Universidade de São Paulo. Contudo, pretendia-se que sua aplicação fosse possível em distintas reservas, para a observação de diferentes animais e que pudesse eventualmente ser expandido para captar demais variáveis de controle relevantes para o biólogo ou veterinário. Finalmente, o sistema obtido cumpre os requisitos definidos e demonstra escalabilidade.

\textbf{Palavras-Chave} -- Monitoramento animal remoto, Redes de sensores sem fio, Sistema embarcado.
\end{resumo}
